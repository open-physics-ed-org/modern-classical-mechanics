\documentclass[11pt]{article}

    \usepackage[breakable]{tcolorbox}
    \usepackage{parskip} % Stop auto-indenting (to mimic markdown behaviour)
    

    % Basic figure setup, for now with no caption control since it's done
    % automatically by Pandoc (which extracts ![](path) syntax from Markdown).
    \usepackage{graphicx}
    % Keep aspect ratio if custom image width or height is specified
    \setkeys{Gin}{keepaspectratio}
    % Maintain compatibility with old templates. Remove in nbconvert 6.0
    \let\Oldincludegraphics\includegraphics
    % Ensure that by default, figures have no caption (until we provide a
    % proper Figure object with a Caption API and a way to capture that
    % in the conversion process - todo).
    \usepackage{caption}
    \DeclareCaptionFormat{nocaption}{}
    \captionsetup{format=nocaption,aboveskip=0pt,belowskip=0pt}

    \usepackage{float}
    \floatplacement{figure}{H} % forces figures to be placed at the correct location
    \usepackage{xcolor} % Allow colors to be defined
    \usepackage{enumerate} % Needed for markdown enumerations to work
    \usepackage{geometry} % Used to adjust the document margins
    \usepackage{amsmath} % Equations
    \usepackage{amssymb} % Equations
    \usepackage{textcomp} % defines textquotesingle
    % Hack from http://tex.stackexchange.com/a/47451/13684:
    \AtBeginDocument{%
        \def\PYZsq{\textquotesingle}% Upright quotes in Pygmentized code
    }
    \usepackage{upquote} % Upright quotes for verbatim code
    \usepackage{eurosym} % defines \euro

    \usepackage{iftex}
    \ifPDFTeX
        \usepackage[T1]{fontenc}
        \IfFileExists{alphabeta.sty}{
              \usepackage{alphabeta}
          }{
              \usepackage[mathletters]{ucs}
              \usepackage[utf8x]{inputenc}
          }
    \else
        \usepackage{fontspec}
        \usepackage{unicode-math}
    \fi

    \usepackage{fancyvrb} % verbatim replacement that allows latex
    \usepackage{grffile} % extends the file name processing of package graphics
                         % to support a larger range
    \makeatletter % fix for old versions of grffile with XeLaTeX
    \@ifpackagelater{grffile}{2019/11/01}
    {
      % Do nothing on new versions
    }
    {
      \def\Gread@@xetex#1{%
        \IfFileExists{"\Gin@base".bb}%
        {\Gread@eps{\Gin@base.bb}}%
        {\Gread@@xetex@aux#1}%
      }
    }
    \makeatother
    \usepackage[Export]{adjustbox} % Used to constrain images to a maximum size
    \adjustboxset{max size={0.9\linewidth}{0.9\paperheight}}

    % The hyperref package gives us a pdf with properly built
    % internal navigation ('pdf bookmarks' for the table of contents,
    % internal cross-reference links, web links for URLs, etc.)
    \usepackage{hyperref}
    % The default LaTeX title has an obnoxious amount of whitespace. By default,
    % titling removes some of it. It also provides customization options.
    \usepackage{titling}
    \usepackage{longtable} % longtable support required by pandoc >1.10
    \usepackage{booktabs}  % table support for pandoc > 1.12.2
    \usepackage{array}     % table support for pandoc >= 2.11.3
    \usepackage{calc}      % table minipage width calculation for pandoc >= 2.11.1
    \usepackage[inline]{enumitem} % IRkernel/repr support (it uses the enumerate* environment)
    \usepackage[normalem]{ulem} % ulem is needed to support strikethroughs (\sout)
                                % normalem makes italics be italics, not underlines
    \usepackage{soul}      % strikethrough (\st) support for pandoc >= 3.0.0
    \usepackage{mathrsfs}
    

    
    % Colors for the hyperref package
    \definecolor{urlcolor}{rgb}{0,.145,.698}
    \definecolor{linkcolor}{rgb}{.71,0.21,0.01}
    \definecolor{citecolor}{rgb}{.12,.54,.11}

    % ANSI colors
    \definecolor{ansi-black}{HTML}{3E424D}
    \definecolor{ansi-black-intense}{HTML}{282C36}
    \definecolor{ansi-red}{HTML}{E75C58}
    \definecolor{ansi-red-intense}{HTML}{B22B31}
    \definecolor{ansi-green}{HTML}{00A250}
    \definecolor{ansi-green-intense}{HTML}{007427}
    \definecolor{ansi-yellow}{HTML}{DDB62B}
    \definecolor{ansi-yellow-intense}{HTML}{B27D12}
    \definecolor{ansi-blue}{HTML}{208FFB}
    \definecolor{ansi-blue-intense}{HTML}{0065CA}
    \definecolor{ansi-magenta}{HTML}{D160C4}
    \definecolor{ansi-magenta-intense}{HTML}{A03196}
    \definecolor{ansi-cyan}{HTML}{60C6C8}
    \definecolor{ansi-cyan-intense}{HTML}{258F8F}
    \definecolor{ansi-white}{HTML}{C5C1B4}
    \definecolor{ansi-white-intense}{HTML}{A1A6B2}
    \definecolor{ansi-default-inverse-fg}{HTML}{FFFFFF}
    \definecolor{ansi-default-inverse-bg}{HTML}{000000}

    % common color for the border for error outputs.
    \definecolor{outerrorbackground}{HTML}{FFDFDF}

    % commands and environments needed by pandoc snippets
    % extracted from the output of `pandoc -s`
    \providecommand{\tightlist}{%
      \setlength{\itemsep}{0pt}\setlength{\parskip}{0pt}}
    \DefineVerbatimEnvironment{Highlighting}{Verbatim}{commandchars=\\\{\}}
    % Add ',fontsize=\small' for more characters per line
    \newenvironment{Shaded}{}{}
    \newcommand{\KeywordTok}[1]{\textcolor[rgb]{0.00,0.44,0.13}{\textbf{{#1}}}}
    \newcommand{\DataTypeTok}[1]{\textcolor[rgb]{0.56,0.13,0.00}{{#1}}}
    \newcommand{\DecValTok}[1]{\textcolor[rgb]{0.25,0.63,0.44}{{#1}}}
    \newcommand{\BaseNTok}[1]{\textcolor[rgb]{0.25,0.63,0.44}{{#1}}}
    \newcommand{\FloatTok}[1]{\textcolor[rgb]{0.25,0.63,0.44}{{#1}}}
    \newcommand{\CharTok}[1]{\textcolor[rgb]{0.25,0.44,0.63}{{#1}}}
    \newcommand{\StringTok}[1]{\textcolor[rgb]{0.25,0.44,0.63}{{#1}}}
    \newcommand{\CommentTok}[1]{\textcolor[rgb]{0.38,0.63,0.69}{\textit{{#1}}}}
    \newcommand{\OtherTok}[1]{\textcolor[rgb]{0.00,0.44,0.13}{{#1}}}
    \newcommand{\AlertTok}[1]{\textcolor[rgb]{1.00,0.00,0.00}{\textbf{{#1}}}}
    \newcommand{\FunctionTok}[1]{\textcolor[rgb]{0.02,0.16,0.49}{{#1}}}
    \newcommand{\RegionMarkerTok}[1]{{#1}}
    \newcommand{\ErrorTok}[1]{\textcolor[rgb]{1.00,0.00,0.00}{\textbf{{#1}}}}
    \newcommand{\NormalTok}[1]{{#1}}

    % Additional commands for more recent versions of Pandoc
    \newcommand{\ConstantTok}[1]{\textcolor[rgb]{0.53,0.00,0.00}{{#1}}}
    \newcommand{\SpecialCharTok}[1]{\textcolor[rgb]{0.25,0.44,0.63}{{#1}}}
    \newcommand{\VerbatimStringTok}[1]{\textcolor[rgb]{0.25,0.44,0.63}{{#1}}}
    \newcommand{\SpecialStringTok}[1]{\textcolor[rgb]{0.73,0.40,0.53}{{#1}}}
    \newcommand{\ImportTok}[1]{{#1}}
    \newcommand{\DocumentationTok}[1]{\textcolor[rgb]{0.73,0.13,0.13}{\textit{{#1}}}}
    \newcommand{\AnnotationTok}[1]{\textcolor[rgb]{0.38,0.63,0.69}{\textbf{\textit{{#1}}}}}
    \newcommand{\CommentVarTok}[1]{\textcolor[rgb]{0.38,0.63,0.69}{\textbf{\textit{{#1}}}}}
    \newcommand{\VariableTok}[1]{\textcolor[rgb]{0.10,0.09,0.49}{{#1}}}
    \newcommand{\ControlFlowTok}[1]{\textcolor[rgb]{0.00,0.44,0.13}{\textbf{{#1}}}}
    \newcommand{\OperatorTok}[1]{\textcolor[rgb]{0.40,0.40,0.40}{{#1}}}
    \newcommand{\BuiltInTok}[1]{{#1}}
    \newcommand{\ExtensionTok}[1]{{#1}}
    \newcommand{\PreprocessorTok}[1]{\textcolor[rgb]{0.74,0.48,0.00}{{#1}}}
    \newcommand{\AttributeTok}[1]{\textcolor[rgb]{0.49,0.56,0.16}{{#1}}}
    \newcommand{\InformationTok}[1]{\textcolor[rgb]{0.38,0.63,0.69}{\textbf{\textit{{#1}}}}}
    \newcommand{\WarningTok}[1]{\textcolor[rgb]{0.38,0.63,0.69}{\textbf{\textit{{#1}}}}}
    \makeatletter
    \newsavebox\pandoc@box
    \newcommand*\pandocbounded[1]{%
      \sbox\pandoc@box{#1}%
      % scaling factors for width and height
      \Gscale@div\@tempa\textheight{\dimexpr\ht\pandoc@box+\dp\pandoc@box\relax}%
      \Gscale@div\@tempb\linewidth{\wd\pandoc@box}%
      % select the smaller of both
      \ifdim\@tempb\p@<\@tempa\p@
        \let\@tempa\@tempb
      \fi
      % scaling accordingly (\@tempa < 1)
      \ifdim\@tempa\p@<\p@
        \scalebox{\@tempa}{\usebox\pandoc@box}%
      % scaling not needed, use as it is
      \else
        \usebox{\pandoc@box}%
      \fi
    }
    \makeatother

    % Define a nice break command that doesn't care if a line doesn't already
    % exist.
    \def\br{\hspace*{\fill} \\* }
    % Math Jax compatibility definitions
    \def\gt{>}
    \def\lt{<}
    \let\Oldtex\TeX
    \let\Oldlatex\LaTeX
    \renewcommand{\TeX}{\textrm{\Oldtex}}
    \renewcommand{\LaTeX}{\textrm{\Oldlatex}}
    % Document parameters
    % Document title
    \title{10\_notes}
    
    
    
    
    
    
    
% Pygments definitions
\makeatletter
\def\PY@reset{\let\PY@it=\relax \let\PY@bf=\relax%
    \let\PY@ul=\relax \let\PY@tc=\relax%
    \let\PY@bc=\relax \let\PY@ff=\relax}
\def\PY@tok#1{\csname PY@tok@#1\endcsname}
\def\PY@toks#1+{\ifx\relax#1\empty\else%
    \PY@tok{#1}\expandafter\PY@toks\fi}
\def\PY@do#1{\PY@bc{\PY@tc{\PY@ul{%
    \PY@it{\PY@bf{\PY@ff{#1}}}}}}}
\def\PY#1#2{\PY@reset\PY@toks#1+\relax+\PY@do{#2}}

\@namedef{PY@tok@w}{\def\PY@tc##1{\textcolor[rgb]{0.73,0.73,0.73}{##1}}}
\@namedef{PY@tok@c}{\let\PY@it=\textit\def\PY@tc##1{\textcolor[rgb]{0.24,0.48,0.48}{##1}}}
\@namedef{PY@tok@cp}{\def\PY@tc##1{\textcolor[rgb]{0.61,0.40,0.00}{##1}}}
\@namedef{PY@tok@k}{\let\PY@bf=\textbf\def\PY@tc##1{\textcolor[rgb]{0.00,0.50,0.00}{##1}}}
\@namedef{PY@tok@kp}{\def\PY@tc##1{\textcolor[rgb]{0.00,0.50,0.00}{##1}}}
\@namedef{PY@tok@kt}{\def\PY@tc##1{\textcolor[rgb]{0.69,0.00,0.25}{##1}}}
\@namedef{PY@tok@o}{\def\PY@tc##1{\textcolor[rgb]{0.40,0.40,0.40}{##1}}}
\@namedef{PY@tok@ow}{\let\PY@bf=\textbf\def\PY@tc##1{\textcolor[rgb]{0.67,0.13,1.00}{##1}}}
\@namedef{PY@tok@nb}{\def\PY@tc##1{\textcolor[rgb]{0.00,0.50,0.00}{##1}}}
\@namedef{PY@tok@nf}{\def\PY@tc##1{\textcolor[rgb]{0.00,0.00,1.00}{##1}}}
\@namedef{PY@tok@nc}{\let\PY@bf=\textbf\def\PY@tc##1{\textcolor[rgb]{0.00,0.00,1.00}{##1}}}
\@namedef{PY@tok@nn}{\let\PY@bf=\textbf\def\PY@tc##1{\textcolor[rgb]{0.00,0.00,1.00}{##1}}}
\@namedef{PY@tok@ne}{\let\PY@bf=\textbf\def\PY@tc##1{\textcolor[rgb]{0.80,0.25,0.22}{##1}}}
\@namedef{PY@tok@nv}{\def\PY@tc##1{\textcolor[rgb]{0.10,0.09,0.49}{##1}}}
\@namedef{PY@tok@no}{\def\PY@tc##1{\textcolor[rgb]{0.53,0.00,0.00}{##1}}}
\@namedef{PY@tok@nl}{\def\PY@tc##1{\textcolor[rgb]{0.46,0.46,0.00}{##1}}}
\@namedef{PY@tok@ni}{\let\PY@bf=\textbf\def\PY@tc##1{\textcolor[rgb]{0.44,0.44,0.44}{##1}}}
\@namedef{PY@tok@na}{\def\PY@tc##1{\textcolor[rgb]{0.41,0.47,0.13}{##1}}}
\@namedef{PY@tok@nt}{\let\PY@bf=\textbf\def\PY@tc##1{\textcolor[rgb]{0.00,0.50,0.00}{##1}}}
\@namedef{PY@tok@nd}{\def\PY@tc##1{\textcolor[rgb]{0.67,0.13,1.00}{##1}}}
\@namedef{PY@tok@s}{\def\PY@tc##1{\textcolor[rgb]{0.73,0.13,0.13}{##1}}}
\@namedef{PY@tok@sd}{\let\PY@it=\textit\def\PY@tc##1{\textcolor[rgb]{0.73,0.13,0.13}{##1}}}
\@namedef{PY@tok@si}{\let\PY@bf=\textbf\def\PY@tc##1{\textcolor[rgb]{0.64,0.35,0.47}{##1}}}
\@namedef{PY@tok@se}{\let\PY@bf=\textbf\def\PY@tc##1{\textcolor[rgb]{0.67,0.36,0.12}{##1}}}
\@namedef{PY@tok@sr}{\def\PY@tc##1{\textcolor[rgb]{0.64,0.35,0.47}{##1}}}
\@namedef{PY@tok@ss}{\def\PY@tc##1{\textcolor[rgb]{0.10,0.09,0.49}{##1}}}
\@namedef{PY@tok@sx}{\def\PY@tc##1{\textcolor[rgb]{0.00,0.50,0.00}{##1}}}
\@namedef{PY@tok@m}{\def\PY@tc##1{\textcolor[rgb]{0.40,0.40,0.40}{##1}}}
\@namedef{PY@tok@gh}{\let\PY@bf=\textbf\def\PY@tc##1{\textcolor[rgb]{0.00,0.00,0.50}{##1}}}
\@namedef{PY@tok@gu}{\let\PY@bf=\textbf\def\PY@tc##1{\textcolor[rgb]{0.50,0.00,0.50}{##1}}}
\@namedef{PY@tok@gd}{\def\PY@tc##1{\textcolor[rgb]{0.63,0.00,0.00}{##1}}}
\@namedef{PY@tok@gi}{\def\PY@tc##1{\textcolor[rgb]{0.00,0.52,0.00}{##1}}}
\@namedef{PY@tok@gr}{\def\PY@tc##1{\textcolor[rgb]{0.89,0.00,0.00}{##1}}}
\@namedef{PY@tok@ge}{\let\PY@it=\textit}
\@namedef{PY@tok@gs}{\let\PY@bf=\textbf}
\@namedef{PY@tok@ges}{\let\PY@bf=\textbf\let\PY@it=\textit}
\@namedef{PY@tok@gp}{\let\PY@bf=\textbf\def\PY@tc##1{\textcolor[rgb]{0.00,0.00,0.50}{##1}}}
\@namedef{PY@tok@go}{\def\PY@tc##1{\textcolor[rgb]{0.44,0.44,0.44}{##1}}}
\@namedef{PY@tok@gt}{\def\PY@tc##1{\textcolor[rgb]{0.00,0.27,0.87}{##1}}}
\@namedef{PY@tok@err}{\def\PY@bc##1{{\setlength{\fboxsep}{\string -\fboxrule}\fcolorbox[rgb]{1.00,0.00,0.00}{1,1,1}{\strut ##1}}}}
\@namedef{PY@tok@kc}{\let\PY@bf=\textbf\def\PY@tc##1{\textcolor[rgb]{0.00,0.50,0.00}{##1}}}
\@namedef{PY@tok@kd}{\let\PY@bf=\textbf\def\PY@tc##1{\textcolor[rgb]{0.00,0.50,0.00}{##1}}}
\@namedef{PY@tok@kn}{\let\PY@bf=\textbf\def\PY@tc##1{\textcolor[rgb]{0.00,0.50,0.00}{##1}}}
\@namedef{PY@tok@kr}{\let\PY@bf=\textbf\def\PY@tc##1{\textcolor[rgb]{0.00,0.50,0.00}{##1}}}
\@namedef{PY@tok@bp}{\def\PY@tc##1{\textcolor[rgb]{0.00,0.50,0.00}{##1}}}
\@namedef{PY@tok@fm}{\def\PY@tc##1{\textcolor[rgb]{0.00,0.00,1.00}{##1}}}
\@namedef{PY@tok@vc}{\def\PY@tc##1{\textcolor[rgb]{0.10,0.09,0.49}{##1}}}
\@namedef{PY@tok@vg}{\def\PY@tc##1{\textcolor[rgb]{0.10,0.09,0.49}{##1}}}
\@namedef{PY@tok@vi}{\def\PY@tc##1{\textcolor[rgb]{0.10,0.09,0.49}{##1}}}
\@namedef{PY@tok@vm}{\def\PY@tc##1{\textcolor[rgb]{0.10,0.09,0.49}{##1}}}
\@namedef{PY@tok@sa}{\def\PY@tc##1{\textcolor[rgb]{0.73,0.13,0.13}{##1}}}
\@namedef{PY@tok@sb}{\def\PY@tc##1{\textcolor[rgb]{0.73,0.13,0.13}{##1}}}
\@namedef{PY@tok@sc}{\def\PY@tc##1{\textcolor[rgb]{0.73,0.13,0.13}{##1}}}
\@namedef{PY@tok@dl}{\def\PY@tc##1{\textcolor[rgb]{0.73,0.13,0.13}{##1}}}
\@namedef{PY@tok@s2}{\def\PY@tc##1{\textcolor[rgb]{0.73,0.13,0.13}{##1}}}
\@namedef{PY@tok@sh}{\def\PY@tc##1{\textcolor[rgb]{0.73,0.13,0.13}{##1}}}
\@namedef{PY@tok@s1}{\def\PY@tc##1{\textcolor[rgb]{0.73,0.13,0.13}{##1}}}
\@namedef{PY@tok@mb}{\def\PY@tc##1{\textcolor[rgb]{0.40,0.40,0.40}{##1}}}
\@namedef{PY@tok@mf}{\def\PY@tc##1{\textcolor[rgb]{0.40,0.40,0.40}{##1}}}
\@namedef{PY@tok@mh}{\def\PY@tc##1{\textcolor[rgb]{0.40,0.40,0.40}{##1}}}
\@namedef{PY@tok@mi}{\def\PY@tc##1{\textcolor[rgb]{0.40,0.40,0.40}{##1}}}
\@namedef{PY@tok@il}{\def\PY@tc##1{\textcolor[rgb]{0.40,0.40,0.40}{##1}}}
\@namedef{PY@tok@mo}{\def\PY@tc##1{\textcolor[rgb]{0.40,0.40,0.40}{##1}}}
\@namedef{PY@tok@ch}{\let\PY@it=\textit\def\PY@tc##1{\textcolor[rgb]{0.24,0.48,0.48}{##1}}}
\@namedef{PY@tok@cm}{\let\PY@it=\textit\def\PY@tc##1{\textcolor[rgb]{0.24,0.48,0.48}{##1}}}
\@namedef{PY@tok@cpf}{\let\PY@it=\textit\def\PY@tc##1{\textcolor[rgb]{0.24,0.48,0.48}{##1}}}
\@namedef{PY@tok@c1}{\let\PY@it=\textit\def\PY@tc##1{\textcolor[rgb]{0.24,0.48,0.48}{##1}}}
\@namedef{PY@tok@cs}{\let\PY@it=\textit\def\PY@tc##1{\textcolor[rgb]{0.24,0.48,0.48}{##1}}}

\def\PYZbs{\char`\\}
\def\PYZus{\char`\_}
\def\PYZob{\char`\{}
\def\PYZcb{\char`\}}
\def\PYZca{\char`\^}
\def\PYZam{\char`\&}
\def\PYZlt{\char`\<}
\def\PYZgt{\char`\>}
\def\PYZsh{\char`\#}
\def\PYZpc{\char`\%}
\def\PYZdl{\char`\$}
\def\PYZhy{\char`\-}
\def\PYZsq{\char`\'}
\def\PYZdq{\char`\"}
\def\PYZti{\char`\~}
% for compatibility with earlier versions
\def\PYZat{@}
\def\PYZlb{[}
\def\PYZrb{]}
\makeatother


    % For linebreaks inside Verbatim environment from package fancyvrb.
    \makeatletter
        \newbox\Wrappedcontinuationbox
        \newbox\Wrappedvisiblespacebox
        \newcommand*\Wrappedvisiblespace {\textcolor{red}{\textvisiblespace}}
        \newcommand*\Wrappedcontinuationsymbol {\textcolor{red}{\llap{\tiny$\m@th\hookrightarrow$}}}
        \newcommand*\Wrappedcontinuationindent {3ex }
        \newcommand*\Wrappedafterbreak {\kern\Wrappedcontinuationindent\copy\Wrappedcontinuationbox}
        % Take advantage of the already applied Pygments mark-up to insert
        % potential linebreaks for TeX processing.
        %        {, <, #, %, $, ' and ": go to next line.
        %        _, }, ^, &, >, - and ~: stay at end of broken line.
        % Use of \textquotesingle for straight quote.
        \newcommand*\Wrappedbreaksatspecials {%
            \def\PYGZus{\discretionary{\char`\_}{\Wrappedafterbreak}{\char`\_}}%
            \def\PYGZob{\discretionary{}{\Wrappedafterbreak\char`\{}{\char`\{}}%
            \def\PYGZcb{\discretionary{\char`\}}{\Wrappedafterbreak}{\char`\}}}%
            \def\PYGZca{\discretionary{\char`\^}{\Wrappedafterbreak}{\char`\^}}%
            \def\PYGZam{\discretionary{\char`\&}{\Wrappedafterbreak}{\char`\&}}%
            \def\PYGZlt{\discretionary{}{\Wrappedafterbreak\char`\<}{\char`\<}}%
            \def\PYGZgt{\discretionary{\char`\>}{\Wrappedafterbreak}{\char`\>}}%
            \def\PYGZsh{\discretionary{}{\Wrappedafterbreak\char`\#}{\char`\#}}%
            \def\PYGZpc{\discretionary{}{\Wrappedafterbreak\char`\%}{\char`\%}}%
            \def\PYGZdl{\discretionary{}{\Wrappedafterbreak\char`\$}{\char`\$}}%
            \def\PYGZhy{\discretionary{\char`\-}{\Wrappedafterbreak}{\char`\-}}%
            \def\PYGZsq{\discretionary{}{\Wrappedafterbreak\textquotesingle}{\textquotesingle}}%
            \def\PYGZdq{\discretionary{}{\Wrappedafterbreak\char`\"}{\char`\"}}%
            \def\PYGZti{\discretionary{\char`\~}{\Wrappedafterbreak}{\char`\~}}%
        }
        % Some characters . , ; ? ! / are not pygmentized.
        % This macro makes them "active" and they will insert potential linebreaks
        \newcommand*\Wrappedbreaksatpunct {%
            \lccode`\~`\.\lowercase{\def~}{\discretionary{\hbox{\char`\.}}{\Wrappedafterbreak}{\hbox{\char`\.}}}%
            \lccode`\~`\,\lowercase{\def~}{\discretionary{\hbox{\char`\,}}{\Wrappedafterbreak}{\hbox{\char`\,}}}%
            \lccode`\~`\;\lowercase{\def~}{\discretionary{\hbox{\char`\;}}{\Wrappedafterbreak}{\hbox{\char`\;}}}%
            \lccode`\~`\:\lowercase{\def~}{\discretionary{\hbox{\char`\:}}{\Wrappedafterbreak}{\hbox{\char`\:}}}%
            \lccode`\~`\?\lowercase{\def~}{\discretionary{\hbox{\char`\?}}{\Wrappedafterbreak}{\hbox{\char`\?}}}%
            \lccode`\~`\!\lowercase{\def~}{\discretionary{\hbox{\char`\!}}{\Wrappedafterbreak}{\hbox{\char`\!}}}%
            \lccode`\~`\/\lowercase{\def~}{\discretionary{\hbox{\char`\/}}{\Wrappedafterbreak}{\hbox{\char`\/}}}%
            \catcode`\.\active
            \catcode`\,\active
            \catcode`\;\active
            \catcode`\:\active
            \catcode`\?\active
            \catcode`\!\active
            \catcode`\/\active
            \lccode`\~`\~
        }
    \makeatother

    \let\OriginalVerbatim=\Verbatim
    \makeatletter
    \renewcommand{\Verbatim}[1][1]{%
        %\parskip\z@skip
        \sbox\Wrappedcontinuationbox {\Wrappedcontinuationsymbol}%
        \sbox\Wrappedvisiblespacebox {\FV@SetupFont\Wrappedvisiblespace}%
        \def\FancyVerbFormatLine ##1{\hsize\linewidth
            \vtop{\raggedright\hyphenpenalty\z@\exhyphenpenalty\z@
                \doublehyphendemerits\z@\finalhyphendemerits\z@
                \strut ##1\strut}%
        }%
        % If the linebreak is at a space, the latter will be displayed as visible
        % space at end of first line, and a continuation symbol starts next line.
        % Stretch/shrink are however usually zero for typewriter font.
        \def\FV@Space {%
            \nobreak\hskip\z@ plus\fontdimen3\font minus\fontdimen4\font
            \discretionary{\copy\Wrappedvisiblespacebox}{\Wrappedafterbreak}
            {\kern\fontdimen2\font}%
        }%

        % Allow breaks at special characters using \PYG... macros.
        \Wrappedbreaksatspecials
        % Breaks at punctuation characters . , ; ? ! and / need catcode=\active
        \OriginalVerbatim[#1,codes*=\Wrappedbreaksatpunct]%
    }
    \makeatother

    % Exact colors from NB
    \definecolor{incolor}{HTML}{303F9F}
    \definecolor{outcolor}{HTML}{D84315}
    \definecolor{cellborder}{HTML}{CFCFCF}
    \definecolor{cellbackground}{HTML}{F7F7F7}

    % prompt
    \makeatletter
    \newcommand{\boxspacing}{\kern\kvtcb@left@rule\kern\kvtcb@boxsep}
    \makeatother
    \newcommand{\prompt}[4]{
        {\ttfamily\llap{{\color{#2}[#3]:\hspace{3pt}#4}}\vspace{-\baselineskip}}
    }
    

    
    % Prevent overflowing lines due to hard-to-break entities
    \sloppy
    % Setup hyperref package
    \hypersetup{
      breaklinks=true,  % so long urls are correctly broken across lines
      colorlinks=true,
      urlcolor=urlcolor,
      linkcolor=linkcolor,
      citecolor=citecolor,
      }
    % Slightly bigger margins than the latex defaults
    
    \geometry{verbose,tmargin=1in,bmargin=1in,lmargin=1in,rmargin=1in}
    
    

\begin{document}
    
    \maketitle
    
    

    
    \section{Week 10 - Activity: Modeling Chaotic
Systems}\label{week-10---activity-modeling-chaotic-systems}

Chaotic systems are complex and unpredictable, often exhibiting
sensitive dependence on initial conditions. This can make numerically
simulating them challenging. We have used a variety of integrators and
noticed that some are better than others for different problems. Here,
we will use the built-in integrator from the \texttt{scipy} library to
simulate our systems
(\href{https://docs.scipy.org/doc/scipy/reference/generated/scipy.integrate.solve_ivp.html}{\texttt{solve\_ivp}}).

This integrator is designed to be robust and efficient, making it a good
choice for a wide range of problems. We can also specify the integrator
to use, which can be helpful for specific applications. In our case, we
will use the default integrator, which is a good general-purpose
integrator. However, we can also specify other integrators, such as
\texttt{RK45}, \texttt{RK23}, \texttt{DOP853}, etc., depending on our
needs. For the most part, these integrators are using different methods
of estimating the slope of the function at each step, which can affect
the accuracy and stability of the simulation. They also have different
efficiency and performance characteristics, which can be important for
large-scale simulations.

We start with a code that simulates a damped driven pendulum, and the
plots we want to obtain. You will can use those to develop simulations
of the \href{https://en.wikipedia.org/wiki/Duffing_equation}{Duffing
oscillator} and the
\href{https://en.wikipedia.org/wiki/Lorenz_system}{Lorenz system} later.

    \subsection{Using solve\_ivp to simulate chaotic
systems}\label{using-solve_ivp-to-simulate-chaotic-systems}

We have written a variety of integrators for solving ordinary
differential equations (ODEs). In this activity, we will introduce the
\texttt{solve\_ivp} function from the \texttt{scipy.integrate} module,
which is a versatile and powerful tool for solving ODEs. We will use it
to simulate chaotic systems and explore its capabilities.

To set up a \texttt{solve\_ivp} simulation, we need to define the system
of ODEs we want to solve. Let's focus on the damped driven pendulum, a
classic example that we will use to illustrate the process.

\subsubsection{Damped Driving Pendulum}\label{damped-driving-pendulum}

Consider a pendulum that can swing in a plane on a pivot. The pendulum
hinge is driven by a horizontal oscillator that oscillates with a
constant frequency and amplitude. The second order ODE that describes
such a damped driven pendulum is:

\[
\frac{d^2\theta}{dt^2} + \beta \frac{d\theta}{dt} + \sin(\theta) = A \cos(\omega_D t)
\]

where \(\theta\) is the angle of the pendulum, \(\beta\) is the damping
coefficient, \(A\) is the amplitude of the driving force, and
\(\omega_D\) is the angular frequency of the driving force. We've
absorbed some constants into the parameters for simplicity.

\subsubsection{Using solve\_ivp}\label{using-solve_ivp}

To use \texttt{solve\_ivp}, we need to convert this second-order ODE
into a system of first-order ODEs. We can do this by introducing a new
variable for the angular velocity:

\[\dot{\theta} = \omega\]
\[\dot{\omega} = -\sin(\theta) - \beta \omega + A \cos(\omega_D t)\]

where \(\omega\) is the angular velocity of the pendulum NOT the driver.

Let's implement this system using \texttt{solve\_ivp} and simulate its
behavior over time.

\textbf{Read over this code and run it.} Make sure you make sense of
what each part does. You will need to modify it for the next part of the
activity.

    \begin{tcolorbox}[breakable, size=fbox, boxrule=1pt, pad at break*=1mm,colback=cellbackground, colframe=cellborder]
\prompt{In}{incolor}{1}{\boxspacing}
\begin{Verbatim}[commandchars=\\\{\}]
\PY{k+kn}{import}\PY{+w}{ }\PY{n+nn}{numpy}\PY{+w}{ }\PY{k}{as}\PY{+w}{ }\PY{n+nn}{np}
\PY{k+kn}{import}\PY{+w}{ }\PY{n+nn}{matplotlib}\PY{n+nn}{.}\PY{n+nn}{pyplot}\PY{+w}{ }\PY{k}{as}\PY{+w}{ }\PY{n+nn}{plt}
\PY{k+kn}{from}\PY{+w}{ }\PY{n+nn}{matplotlib}\PY{+w}{ }\PY{k+kn}{import} \PY{n}{gridspec}
\PY{k+kn}{from}\PY{+w}{ }\PY{n+nn}{scipy}\PY{n+nn}{.}\PY{n+nn}{integrate}\PY{+w}{ }\PY{k+kn}{import} \PY{n}{solve\PYZus{}ivp}
\PY{n}{plt}\PY{o}{.}\PY{n}{style}\PY{o}{.}\PY{n}{use}\PY{p}{(}\PY{l+s+s1}{\PYZsq{}}\PY{l+s+s1}{seaborn\PYZhy{}v0\PYZus{}8\PYZhy{}colorblind}\PY{l+s+s1}{\PYZsq{}}\PY{p}{)}

\PY{c+c1}{\PYZsh{} Define the damped driven pendulum equations}
\PY{c+c1}{\PYZsh{} Note that they are first order ODEs}
\PY{k}{def}\PY{+w}{ }\PY{n+nf}{damped\PYZus{}driven\PYZus{}pendulum}\PY{p}{(}\PY{n}{t}\PY{p}{,} \PY{n}{y}\PY{p}{,} \PY{n}{beta}\PY{p}{,} \PY{n}{A}\PY{p}{,} \PY{n}{omegaD}\PY{o}{=}\PY{l+m+mi}{1}\PY{p}{)}\PY{p}{:}
    \PY{n}{theta}\PY{p}{,} \PY{n}{omega} \PY{o}{=} \PY{n}{y}
    \PY{n}{dtheta\PYZus{}dt} \PY{o}{=} \PY{n}{omega}
    \PY{n}{domega\PYZus{}dt} \PY{o}{=} \PY{o}{\PYZhy{}}\PY{n}{np}\PY{o}{.}\PY{n}{sin}\PY{p}{(}\PY{n}{theta}\PY{p}{)} \PY{o}{\PYZhy{}} \PY{n}{beta} \PY{o}{*} \PY{n}{omega} \PY{o}{+} \PY{n}{A} \PY{o}{*} \PY{n}{np}\PY{o}{.}\PY{n}{cos}\PY{p}{(}\PY{n}{omegaD}\PY{o}{*}\PY{n}{t}\PY{p}{)}
    \PY{k}{return} \PY{p}{[}\PY{n}{dtheta\PYZus{}dt}\PY{p}{,} \PY{n}{domega\PYZus{}dt}\PY{p}{]}

\PY{c+c1}{\PYZsh{} Parameters that define the system}
\PY{n}{beta} \PY{o}{=} \PY{l+m+mf}{0.5}         \PY{c+c1}{\PYZsh{} Damping coefficient}
\PY{n}{A} \PY{o}{=} \PY{l+m+mf}{1.0}             \PY{c+c1}{\PYZsh{} Amplitude of driving force}
\PY{n}{omegaD} \PY{o}{=} \PY{l+m+mi}{2}\PY{o}{*}\PY{n}{np}\PY{o}{.}\PY{n}{pi}  \PY{c+c1}{\PYZsh{} Driving frequency}
\PY{n}{t\PYZus{}span} \PY{o}{=} \PY{p}{(}\PY{l+m+mi}{0}\PY{p}{,} \PY{l+m+mi}{100}\PY{p}{)}   \PY{c+c1}{\PYZsh{} Time span for the simulation}
\PY{n}{y0} \PY{o}{=} \PY{p}{[}\PY{l+m+mi}{6}\PY{p}{,} \PY{l+m+mi}{0}\PY{p}{]}     \PY{c+c1}{\PYZsh{} Initial conditions: theta = pi, omega = 0; Note that solve\PYZus{}ivp takes a list of initial conditions}
\PY{n}{t\PYZus{}eval} \PY{o}{=} \PY{n}{np}\PY{o}{.}\PY{n}{linspace}\PY{p}{(}\PY{n}{t\PYZus{}span}\PY{p}{[}\PY{l+m+mi}{0}\PY{p}{]}\PY{p}{,} \PY{n}{t\PYZus{}span}\PY{p}{[}\PY{l+m+mi}{1}\PY{p}{]}\PY{p}{,} \PY{l+m+mi}{10000}\PY{p}{)}  \PY{c+c1}{\PYZsh{} Time points to evaluate; Note that solve\PYZus{}ivp takes a list of times to evaluate}

\PY{c+c1}{\PYZsh{} Solve the ODEs using solve\PYZus{}ivp}
\PY{c+c1}{\PYZsh{} Here we pass the function, time span, initial conditions, parameters, and time points to evaluate}
\PY{c+c1}{\PYZsh{} Any additional arguments to the function can be passed in args in the form of a tuple}
\PY{c+c1}{\PYZsh{} Note that the order of arguments in args must match the order of parameters in the function}
\PY{n}{solution} \PY{o}{=} \PY{n}{solve\PYZus{}ivp}\PY{p}{(}\PY{n}{damped\PYZus{}driven\PYZus{}pendulum}\PY{p}{,} \PY{n}{t\PYZus{}span}\PY{p}{,} \PY{n}{y0}\PY{p}{,} \PY{n}{args}\PY{o}{=}\PY{p}{(}\PY{n}{beta}\PY{p}{,} \PY{n}{A}\PY{p}{,} \PY{n}{omegaD}\PY{p}{)}\PY{p}{,} \PY{n}{t\PYZus{}eval}\PY{o}{=}\PY{n}{t\PYZus{}eval}\PY{p}{)}
\end{Verbatim}
\end{tcolorbox}

    \subsubsection{Plotting the resulting
solution}\label{plotting-the-resulting-solution}

The solution from \texttt{solve\_ivp} gives us the values of \(\theta\)
and \(\omega\) over time. We can use this data to create plots that help
us visualize the behavior of the damped driven pendulum. These results
are evaluated at discrete time points, so we can use them to create time
series plots or phase space plots.

The solution object returned by \texttt{solve\_ivp} contains several
attributes, including \texttt{t} (the time points) and \texttt{y} (the
values of the variables at each time point). We can use these to create
our plots. Note that the first column of \texttt{y} corresponds to
\(\theta\) and the second column corresponds to \(\omega\) because we
defined them in that order in the \texttt{damped\_driven\_pendulum}
function above.

    \begin{tcolorbox}[breakable, size=fbox, boxrule=1pt, pad at break*=1mm,colback=cellbackground, colframe=cellborder]
\prompt{In}{incolor}{2}{\boxspacing}
\begin{Verbatim}[commandchars=\\\{\}]
\PY{n}{fig} \PY{o}{=} \PY{n}{plt}\PY{o}{.}\PY{n}{figure}\PY{p}{(}\PY{n}{figsize}\PY{o}{=}\PY{p}{(}\PY{l+m+mi}{16}\PY{p}{,} \PY{l+m+mi}{4}\PY{p}{)}\PY{p}{)}
\PY{n}{gs} \PY{o}{=} \PY{n}{gridspec}\PY{o}{.}\PY{n}{GridSpec}\PY{p}{(}\PY{l+m+mi}{2}\PY{p}{,} \PY{l+m+mi}{2}\PY{p}{,} \PY{n}{figure}\PY{o}{=}\PY{n}{fig}\PY{p}{,} \PY{n}{height\PYZus{}ratios}\PY{o}{=}\PY{p}{[}\PY{l+m+mi}{1}\PY{p}{,} \PY{l+m+mi}{1}\PY{p}{]}\PY{p}{,} \PY{n}{width\PYZus{}ratios}\PY{o}{=}\PY{p}{[}\PY{l+m+mi}{3}\PY{p}{,} \PY{l+m+mi}{1}\PY{p}{]}\PY{p}{)}


\PY{n}{ax0} \PY{o}{=} \PY{n}{fig}\PY{o}{.}\PY{n}{add\PYZus{}subplot}\PY{p}{(}\PY{n}{gs}\PY{p}{[}\PY{l+m+mi}{0}\PY{p}{,} \PY{l+m+mi}{0}\PY{p}{]}\PY{p}{)}
\PY{n}{ax0}\PY{o}{.}\PY{n}{plot}\PY{p}{(}\PY{n}{solution}\PY{o}{.}\PY{n}{t}\PY{p}{,} \PY{n}{solution}\PY{o}{.}\PY{n}{y}\PY{p}{[}\PY{l+m+mi}{0}\PY{p}{]}\PY{p}{)}
\PY{n}{plt}\PY{o}{.}\PY{n}{scatter}\PY{p}{(}\PY{n}{solution}\PY{o}{.}\PY{n}{t}\PY{p}{[}\PY{l+m+mi}{0}\PY{p}{]}\PY{p}{,} \PY{n}{solution}\PY{o}{.}\PY{n}{y}\PY{p}{[}\PY{l+m+mi}{0}\PY{p}{]}\PY{p}{[}\PY{l+m+mi}{0}\PY{p}{]}\PY{p}{,} \PY{n}{color}\PY{o}{=}\PY{l+s+s1}{\PYZsq{}}\PY{l+s+s1}{C1}\PY{l+s+s1}{\PYZsq{}}\PY{p}{,} \PY{n}{marker}\PY{o}{=}\PY{l+s+s1}{\PYZsq{}}\PY{l+s+s1}{o}\PY{l+s+s1}{\PYZsq{}}\PY{p}{,} \PY{n}{label}\PY{o}{=}\PY{l+s+s1}{\PYZsq{}}\PY{l+s+s1}{Start}\PY{l+s+s1}{\PYZsq{}}\PY{p}{,} \PY{n}{s}\PY{o}{=}\PY{l+m+mi}{50}\PY{p}{)}
\PY{n}{plt}\PY{o}{.}\PY{n}{scatter}\PY{p}{(}\PY{n}{solution}\PY{o}{.}\PY{n}{t}\PY{p}{[}\PY{o}{\PYZhy{}}\PY{l+m+mi}{1}\PY{p}{]}\PY{p}{,} \PY{n}{solution}\PY{o}{.}\PY{n}{y}\PY{p}{[}\PY{l+m+mi}{0}\PY{p}{]}\PY{p}{[}\PY{o}{\PYZhy{}}\PY{l+m+mi}{1}\PY{p}{]}\PY{p}{,} \PY{n}{color}\PY{o}{=}\PY{l+s+s1}{\PYZsq{}}\PY{l+s+s1}{C2}\PY{l+s+s1}{\PYZsq{}}\PY{p}{,} \PY{n}{marker}\PY{o}{=}\PY{l+s+s1}{\PYZsq{}}\PY{l+s+s1}{s}\PY{l+s+s1}{\PYZsq{}}\PY{p}{,} \PY{n}{label}\PY{o}{=}\PY{l+s+s1}{\PYZsq{}}\PY{l+s+s1}{End}\PY{l+s+s1}{\PYZsq{}}\PY{p}{,} \PY{n}{s}\PY{o}{=}\PY{l+m+mi}{50}\PY{p}{)}
\PY{n}{ax0}\PY{o}{.}\PY{n}{set\PYZus{}title}\PY{p}{(}\PY{l+s+s1}{\PYZsq{}}\PY{l+s+s1}{Time Series}\PY{l+s+s1}{\PYZsq{}}\PY{p}{)}
\PY{n}{ax0}\PY{o}{.}\PY{n}{set\PYZus{}ylabel}\PY{p}{(}\PY{l+s+s1}{\PYZsq{}}\PY{l+s+s1}{Position}\PY{l+s+s1}{\PYZsq{}}\PY{p}{)}
\PY{n}{ax0}\PY{o}{.}\PY{n}{legend}\PY{p}{(}\PY{p}{)}
\PY{n}{ax0}\PY{o}{.}\PY{n}{grid}\PY{p}{(}\PY{p}{)}

\PY{n}{ax1} \PY{o}{=} \PY{n}{fig}\PY{o}{.}\PY{n}{add\PYZus{}subplot}\PY{p}{(}\PY{n}{gs}\PY{p}{[}\PY{l+m+mi}{1}\PY{p}{,}\PY{l+m+mi}{0}\PY{p}{]}\PY{p}{)}
\PY{n}{ax1}\PY{o}{.}\PY{n}{plot}\PY{p}{(}\PY{n}{solution}\PY{o}{.}\PY{n}{t}\PY{p}{,} \PY{n}{solution}\PY{o}{.}\PY{n}{y}\PY{p}{[}\PY{l+m+mi}{1}\PY{p}{]}\PY{p}{)}
\PY{n}{plt}\PY{o}{.}\PY{n}{scatter}\PY{p}{(}\PY{n}{solution}\PY{o}{.}\PY{n}{t}\PY{p}{[}\PY{l+m+mi}{0}\PY{p}{]}\PY{p}{,} \PY{n}{solution}\PY{o}{.}\PY{n}{y}\PY{p}{[}\PY{l+m+mi}{1}\PY{p}{]}\PY{p}{[}\PY{l+m+mi}{0}\PY{p}{]}\PY{p}{,} \PY{n}{color}\PY{o}{=}\PY{l+s+s1}{\PYZsq{}}\PY{l+s+s1}{C1}\PY{l+s+s1}{\PYZsq{}}\PY{p}{,} \PY{n}{marker}\PY{o}{=}\PY{l+s+s1}{\PYZsq{}}\PY{l+s+s1}{o}\PY{l+s+s1}{\PYZsq{}}\PY{p}{,} \PY{n}{label}\PY{o}{=}\PY{l+s+s1}{\PYZsq{}}\PY{l+s+s1}{Start}\PY{l+s+s1}{\PYZsq{}}\PY{p}{,} \PY{n}{s}\PY{o}{=}\PY{l+m+mi}{50}\PY{p}{)}
\PY{n}{plt}\PY{o}{.}\PY{n}{scatter}\PY{p}{(}\PY{n}{solution}\PY{o}{.}\PY{n}{t}\PY{p}{[}\PY{o}{\PYZhy{}}\PY{l+m+mi}{1}\PY{p}{]}\PY{p}{,} \PY{n}{solution}\PY{o}{.}\PY{n}{y}\PY{p}{[}\PY{l+m+mi}{1}\PY{p}{]}\PY{p}{[}\PY{o}{\PYZhy{}}\PY{l+m+mi}{1}\PY{p}{]}\PY{p}{,} \PY{n}{color}\PY{o}{=}\PY{l+s+s1}{\PYZsq{}}\PY{l+s+s1}{C2}\PY{l+s+s1}{\PYZsq{}}\PY{p}{,} \PY{n}{marker}\PY{o}{=}\PY{l+s+s1}{\PYZsq{}}\PY{l+s+s1}{s}\PY{l+s+s1}{\PYZsq{}}\PY{p}{,} \PY{n}{label}\PY{o}{=}\PY{l+s+s1}{\PYZsq{}}\PY{l+s+s1}{End}\PY{l+s+s1}{\PYZsq{}}\PY{p}{,} \PY{n}{s}\PY{o}{=}\PY{l+m+mi}{50}\PY{p}{)}
\PY{n}{ax1}\PY{o}{.}\PY{n}{set\PYZus{}xlabel}\PY{p}{(}\PY{l+s+s1}{\PYZsq{}}\PY{l+s+s1}{Time}\PY{l+s+s1}{\PYZsq{}}\PY{p}{)}
\PY{n}{ax1}\PY{o}{.}\PY{n}{set\PYZus{}ylabel}\PY{p}{(}\PY{l+s+s1}{\PYZsq{}}\PY{l+s+s1}{Velocity}\PY{l+s+s1}{\PYZsq{}}\PY{p}{)}
\PY{n}{ax1}\PY{o}{.}\PY{n}{legend}\PY{p}{(}\PY{p}{)}
\PY{n}{ax1}\PY{o}{.}\PY{n}{grid}\PY{p}{(}\PY{p}{)}

\PY{n}{ax2} \PY{o}{=} \PY{n}{fig}\PY{o}{.}\PY{n}{add\PYZus{}subplot}\PY{p}{(}\PY{n}{gs}\PY{p}{[}\PY{p}{:}\PY{p}{,} \PY{l+m+mi}{1}\PY{p}{]}\PY{p}{)}
\PY{n}{ax2}\PY{o}{.}\PY{n}{plot}\PY{p}{(}\PY{n}{solution}\PY{o}{.}\PY{n}{y}\PY{p}{[}\PY{l+m+mi}{0}\PY{p}{]}\PY{p}{,} \PY{n}{solution}\PY{o}{.}\PY{n}{y}\PY{p}{[}\PY{l+m+mi}{1}\PY{p}{]}\PY{p}{)}
\PY{n}{ax2}\PY{o}{.}\PY{n}{set\PYZus{}title}\PY{p}{(}\PY{l+s+s1}{\PYZsq{}}\PY{l+s+s1}{Phase Space}\PY{l+s+s1}{\PYZsq{}}\PY{p}{)}
\PY{n}{plt}\PY{o}{.}\PY{n}{scatter}\PY{p}{(}\PY{n}{solution}\PY{o}{.}\PY{n}{y}\PY{p}{[}\PY{l+m+mi}{0}\PY{p}{]}\PY{p}{[}\PY{l+m+mi}{0}\PY{p}{]}\PY{p}{,} \PY{n}{solution}\PY{o}{.}\PY{n}{y}\PY{p}{[}\PY{l+m+mi}{1}\PY{p}{]}\PY{p}{[}\PY{l+m+mi}{0}\PY{p}{]}\PY{p}{,} \PY{n}{color}\PY{o}{=}\PY{l+s+s1}{\PYZsq{}}\PY{l+s+s1}{C1}\PY{l+s+s1}{\PYZsq{}}\PY{p}{,} \PY{n}{marker}\PY{o}{=}\PY{l+s+s1}{\PYZsq{}}\PY{l+s+s1}{o}\PY{l+s+s1}{\PYZsq{}}\PY{p}{,} \PY{n}{label}\PY{o}{=}\PY{l+s+s1}{\PYZsq{}}\PY{l+s+s1}{Start}\PY{l+s+s1}{\PYZsq{}}\PY{p}{,} \PY{n}{s}\PY{o}{=}\PY{l+m+mi}{50}\PY{p}{)}
\PY{n}{plt}\PY{o}{.}\PY{n}{scatter}\PY{p}{(}\PY{n}{solution}\PY{o}{.}\PY{n}{y}\PY{p}{[}\PY{l+m+mi}{0}\PY{p}{]}\PY{p}{[}\PY{o}{\PYZhy{}}\PY{l+m+mi}{1}\PY{p}{]}\PY{p}{,} \PY{n}{solution}\PY{o}{.}\PY{n}{y}\PY{p}{[}\PY{l+m+mi}{1}\PY{p}{]}\PY{p}{[}\PY{o}{\PYZhy{}}\PY{l+m+mi}{1}\PY{p}{]}\PY{p}{,} \PY{n}{color}\PY{o}{=}\PY{l+s+s1}{\PYZsq{}}\PY{l+s+s1}{C2}\PY{l+s+s1}{\PYZsq{}}\PY{p}{,} \PY{n}{marker}\PY{o}{=}\PY{l+s+s1}{\PYZsq{}}\PY{l+s+s1}{s}\PY{l+s+s1}{\PYZsq{}}\PY{p}{,} \PY{n}{label}\PY{o}{=}\PY{l+s+s1}{\PYZsq{}}\PY{l+s+s1}{End}\PY{l+s+s1}{\PYZsq{}}\PY{p}{,} \PY{n}{s}\PY{o}{=}\PY{l+m+mi}{50}\PY{p}{)}
\PY{n}{ax2}\PY{o}{.}\PY{n}{set\PYZus{}xlabel}\PY{p}{(}\PY{l+s+s1}{\PYZsq{}}\PY{l+s+s1}{Position}\PY{l+s+s1}{\PYZsq{}}\PY{p}{)}
\PY{n}{ax2}\PY{o}{.}\PY{n}{set\PYZus{}ylabel}\PY{p}{(}\PY{l+s+s1}{\PYZsq{}}\PY{l+s+s1}{Velocity}\PY{l+s+s1}{\PYZsq{}}\PY{p}{)}
\PY{n}{ax2}\PY{o}{.}\PY{n}{legend}\PY{p}{(}\PY{p}{)}
\PY{n}{ax2}\PY{o}{.}\PY{n}{grid}\PY{p}{(}\PY{p}{)}
\PY{n}{plt}\PY{o}{.}\PY{n}{tight\PYZus{}layout}\PY{p}{(}\PY{p}{)}
\end{Verbatim}
\end{tcolorbox}

    \begin{center}
    \adjustimage{max size={0.9\linewidth}{0.9\paperheight}}{10_notes_files/10_notes_4_0.png}
    \end{center}
    { \hspace*{\fill} \\}
    
    What do you notice about the behavior of this system? Both short and
long term? How can you see that in your plots?

\textbf{Your Answer Here}

    \subsubsection{Investigating
Periodicity}\label{investigating-periodicity}

As you recall, the damped driven oscillator has two solutions - a
transient solution that decays to zero, and a steady-state solution that
is periodic. You can see in the above plot that the system starts with a
transient solution, but eventually settles into a periodic solution.
Let's truncate the time to see the periodic solution more clearly. Let's
look at the last 40 seconds of the simulation. We used \texttt{t\_eval}
to only evaluate the solution at 1000 points, between 0 and 100. We
don't need to recalculate the solution, we can just plot the last 40
seconds of the solution we already calculated.

To check the periodicity, we investigate if the time series returns to
the same value after a certain period. We can do this by plotting the
time series after the transients have decayed. If the system is
periodic, we should see a repeating pattern in the plot. Using the
physics of the driver, we can plot a point at every period of the driver
to see if the system returns to the same value,
\(T = \frac{2\pi}{\omega}\).

We do that below to demonstrate that after the transient solution
decays, the system settles into a single periodic solution.

One route to chaos is
\href{https://en.wikipedia.org/wiki/Period-doubling_bifurcation}{period-doubling
bifurcation}, where the system goes from a single period to two periods.
You can see that we are not at that point yet, but the
\href{https://en.wikipedia.org/wiki/Duffing_equation}{Duffing
oscillator} is a classic example of a system that exhibits
period-doubling bifurcation.

    \begin{tcolorbox}[breakable, size=fbox, boxrule=1pt, pad at break*=1mm,colback=cellbackground, colframe=cellborder]
\prompt{In}{incolor}{3}{\boxspacing}
\begin{Verbatim}[commandchars=\\\{\}]
\PY{c+c1}{\PYZsh{} Note this block runs the integration again but only returns the last 60 seconds of the simulation}
\PY{n}{startTime} \PY{o}{=} \PY{l+m+mi}{60}
\PY{n}{index} \PY{o}{=} \PY{n}{np}\PY{o}{.}\PY{n}{where}\PY{p}{(}\PY{n}{solution}\PY{o}{.}\PY{n}{t} \PY{o}{\PYZgt{}} \PY{n}{startTime}\PY{p}{)}\PY{p}{[}\PY{l+m+mi}{0}\PY{p}{]}\PY{p}{[}\PY{l+m+mi}{0}\PY{p}{]}

\PY{n}{solution\PYZus{}short\PYZus{}time} \PY{o}{=} \PY{n}{solution}\PY{o}{.}\PY{n}{t}\PY{p}{[}\PY{n}{index}\PY{p}{:}\PY{p}{]}
\PY{n}{solution\PYZus{}short\PYZus{}theta} \PY{o}{=} \PY{n}{solution}\PY{o}{.}\PY{n}{y}\PY{p}{[}\PY{l+m+mi}{0}\PY{p}{]}\PY{p}{[}\PY{n}{index}\PY{p}{:}\PY{p}{]}
\PY{n}{solution\PYZus{}short\PYZus{}omega} \PY{o}{=} \PY{n}{solution}\PY{o}{.}\PY{n}{y}\PY{p}{[}\PY{l+m+mi}{1}\PY{p}{]}\PY{p}{[}\PY{n}{index}\PY{p}{:}\PY{p}{]}

\PY{c+c1}{\PYZsh{} Indicies where solution repeats every 2pi/omegaD}
\PY{n}{period} \PY{o}{=} \PY{l+m+mi}{2}\PY{o}{*}\PY{n}{np}\PY{o}{.}\PY{n}{pi}\PY{o}{/}\PY{n}{omegaD}
\PY{n}{indices} \PY{o}{=} \PY{n}{np}\PY{o}{.}\PY{n}{where}\PY{p}{(}\PY{p}{(}\PY{n}{solution\PYZus{}short\PYZus{}time} \PY{o}{\PYZpc{}} \PY{n}{period}\PY{p}{)} \PY{o}{\PYZlt{}} \PY{l+m+mf}{0.01}\PY{p}{)}\PY{p}{[}\PY{l+m+mi}{0}\PY{p}{]}


\PY{n}{fig} \PY{o}{=} \PY{n}{plt}\PY{o}{.}\PY{n}{figure}\PY{p}{(}\PY{n}{figsize}\PY{o}{=}\PY{p}{(}\PY{l+m+mi}{16}\PY{p}{,} \PY{l+m+mi}{4}\PY{p}{)}\PY{p}{)}
\PY{n}{gs} \PY{o}{=} \PY{n}{gridspec}\PY{o}{.}\PY{n}{GridSpec}\PY{p}{(}\PY{l+m+mi}{2}\PY{p}{,} \PY{l+m+mi}{2}\PY{p}{,} \PY{n}{figure}\PY{o}{=}\PY{n}{fig}\PY{p}{,} \PY{n}{height\PYZus{}ratios}\PY{o}{=}\PY{p}{[}\PY{l+m+mi}{1}\PY{p}{,} \PY{l+m+mi}{1}\PY{p}{]}\PY{p}{,} \PY{n}{width\PYZus{}ratios}\PY{o}{=}\PY{p}{[}\PY{l+m+mi}{3}\PY{p}{,} \PY{l+m+mi}{1}\PY{p}{]}\PY{p}{)}


\PY{n}{ax0} \PY{o}{=} \PY{n}{fig}\PY{o}{.}\PY{n}{add\PYZus{}subplot}\PY{p}{(}\PY{n}{gs}\PY{p}{[}\PY{l+m+mi}{0}\PY{p}{,} \PY{l+m+mi}{0}\PY{p}{]}\PY{p}{)}
\PY{n}{ax0}\PY{o}{.}\PY{n}{plot}\PY{p}{(}\PY{n}{solution\PYZus{}short\PYZus{}time}\PY{p}{,} \PY{n}{solution\PYZus{}short\PYZus{}theta}\PY{p}{)}
\PY{n}{ax0}\PY{o}{.}\PY{n}{scatter}\PY{p}{(}\PY{n}{solution\PYZus{}short\PYZus{}time}\PY{p}{[}\PY{l+m+mi}{0}\PY{p}{]}\PY{p}{,} \PY{n}{solution\PYZus{}short\PYZus{}theta}\PY{p}{[}\PY{l+m+mi}{0}\PY{p}{]}\PY{p}{,} \PY{n}{color}\PY{o}{=}\PY{l+s+s1}{\PYZsq{}}\PY{l+s+s1}{C1}\PY{l+s+s1}{\PYZsq{}}\PY{p}{,} \PY{n}{marker}\PY{o}{=}\PY{l+s+s1}{\PYZsq{}}\PY{l+s+s1}{o}\PY{l+s+s1}{\PYZsq{}}\PY{p}{,} \PY{n}{label}\PY{o}{=}\PY{l+s+s1}{\PYZsq{}}\PY{l+s+s1}{Start}\PY{l+s+s1}{\PYZsq{}}\PY{p}{,} \PY{n}{s}\PY{o}{=}\PY{l+m+mi}{50}\PY{p}{)}
\PY{n}{ax0}\PY{o}{.}\PY{n}{scatter}\PY{p}{(}\PY{n}{solution\PYZus{}short\PYZus{}time}\PY{p}{[}\PY{o}{\PYZhy{}}\PY{l+m+mi}{1}\PY{p}{]}\PY{p}{,} \PY{n}{solution\PYZus{}short\PYZus{}theta}\PY{p}{[}\PY{o}{\PYZhy{}}\PY{l+m+mi}{1}\PY{p}{]}\PY{p}{,} \PY{n}{color}\PY{o}{=}\PY{l+s+s1}{\PYZsq{}}\PY{l+s+s1}{C2}\PY{l+s+s1}{\PYZsq{}}\PY{p}{,} \PY{n}{marker}\PY{o}{=}\PY{l+s+s1}{\PYZsq{}}\PY{l+s+s1}{s}\PY{l+s+s1}{\PYZsq{}}\PY{p}{,} \PY{n}{label}\PY{o}{=}\PY{l+s+s1}{\PYZsq{}}\PY{l+s+s1}{End}\PY{l+s+s1}{\PYZsq{}}\PY{p}{,} \PY{n}{s}\PY{o}{=}\PY{l+m+mi}{50}\PY{p}{)}
\PY{n}{ax0}\PY{o}{.}\PY{n}{scatter}\PY{p}{(}\PY{n}{solution\PYZus{}short\PYZus{}time}\PY{p}{[}\PY{n}{indices}\PY{p}{]}\PY{p}{,} \PY{n}{solution\PYZus{}short\PYZus{}theta}\PY{p}{[}\PY{n}{indices}\PY{p}{]}\PY{p}{,} \PY{n}{color}\PY{o}{=}\PY{l+s+s1}{\PYZsq{}}\PY{l+s+s1}{C3}\PY{l+s+s1}{\PYZsq{}}\PY{p}{,} \PY{n}{marker}\PY{o}{=}\PY{l+s+s1}{\PYZsq{}}\PY{l+s+s1}{x}\PY{l+s+s1}{\PYZsq{}}\PY{p}{,} \PY{n}{s}\PY{o}{=}\PY{l+m+mi}{50}\PY{p}{)}
\PY{n}{ax0}\PY{o}{.}\PY{n}{set\PYZus{}title}\PY{p}{(}\PY{l+s+s1}{\PYZsq{}}\PY{l+s+s1}{Time Series}\PY{l+s+s1}{\PYZsq{}}\PY{p}{)}
\PY{n}{ax0}\PY{o}{.}\PY{n}{set\PYZus{}ylabel}\PY{p}{(}\PY{l+s+s1}{\PYZsq{}}\PY{l+s+s1}{Position}\PY{l+s+s1}{\PYZsq{}}\PY{p}{)}
\PY{n}{ax0}\PY{o}{.}\PY{n}{legend}\PY{p}{(}\PY{p}{)}
\PY{n}{ax0}\PY{o}{.}\PY{n}{grid}\PY{p}{(}\PY{p}{)}

\PY{n}{ax1} \PY{o}{=} \PY{n}{fig}\PY{o}{.}\PY{n}{add\PYZus{}subplot}\PY{p}{(}\PY{n}{gs}\PY{p}{[}\PY{l+m+mi}{1}\PY{p}{,}\PY{l+m+mi}{0}\PY{p}{]}\PY{p}{)}
\PY{n}{ax1}\PY{o}{.}\PY{n}{plot}\PY{p}{(}\PY{n}{solution\PYZus{}short\PYZus{}time}\PY{p}{,} \PY{n}{solution\PYZus{}short\PYZus{}omega}\PY{p}{)}
\PY{n}{ax1}\PY{o}{.}\PY{n}{scatter}\PY{p}{(}\PY{n}{solution\PYZus{}short\PYZus{}time}\PY{p}{[}\PY{l+m+mi}{0}\PY{p}{]}\PY{p}{,} \PY{n}{solution\PYZus{}short\PYZus{}omega}\PY{p}{[}\PY{l+m+mi}{0}\PY{p}{]}\PY{p}{,} \PY{n}{color}\PY{o}{=}\PY{l+s+s1}{\PYZsq{}}\PY{l+s+s1}{C1}\PY{l+s+s1}{\PYZsq{}}\PY{p}{,} \PY{n}{marker}\PY{o}{=}\PY{l+s+s1}{\PYZsq{}}\PY{l+s+s1}{o}\PY{l+s+s1}{\PYZsq{}}\PY{p}{,} \PY{n}{label}\PY{o}{=}\PY{l+s+s1}{\PYZsq{}}\PY{l+s+s1}{Start}\PY{l+s+s1}{\PYZsq{}}\PY{p}{,} \PY{n}{s}\PY{o}{=}\PY{l+m+mi}{50}\PY{p}{)}
\PY{n}{ax1}\PY{o}{.}\PY{n}{scatter}\PY{p}{(}\PY{n}{solution\PYZus{}short\PYZus{}time}\PY{p}{[}\PY{o}{\PYZhy{}}\PY{l+m+mi}{1}\PY{p}{]}\PY{p}{,} \PY{n}{solution\PYZus{}short\PYZus{}omega}\PY{p}{[}\PY{o}{\PYZhy{}}\PY{l+m+mi}{1}\PY{p}{]}\PY{p}{,} \PY{n}{color}\PY{o}{=}\PY{l+s+s1}{\PYZsq{}}\PY{l+s+s1}{C2}\PY{l+s+s1}{\PYZsq{}}\PY{p}{,} \PY{n}{marker}\PY{o}{=}\PY{l+s+s1}{\PYZsq{}}\PY{l+s+s1}{s}\PY{l+s+s1}{\PYZsq{}}\PY{p}{,} \PY{n}{label}\PY{o}{=}\PY{l+s+s1}{\PYZsq{}}\PY{l+s+s1}{End}\PY{l+s+s1}{\PYZsq{}}\PY{p}{,} \PY{n}{s}\PY{o}{=}\PY{l+m+mi}{50}\PY{p}{)}
\PY{n}{ax1}\PY{o}{.}\PY{n}{scatter}\PY{p}{(}\PY{n}{solution\PYZus{}short\PYZus{}time}\PY{p}{[}\PY{n}{indices}\PY{p}{]}\PY{p}{,} \PY{n}{solution\PYZus{}short\PYZus{}omega}\PY{p}{[}\PY{n}{indices}\PY{p}{]}\PY{p}{,} \PY{n}{color}\PY{o}{=}\PY{l+s+s1}{\PYZsq{}}\PY{l+s+s1}{C3}\PY{l+s+s1}{\PYZsq{}}\PY{p}{,} \PY{n}{marker}\PY{o}{=}\PY{l+s+s1}{\PYZsq{}}\PY{l+s+s1}{x}\PY{l+s+s1}{\PYZsq{}}\PY{p}{,} \PY{n}{s}\PY{o}{=}\PY{l+m+mi}{50}\PY{p}{)}
\PY{n}{ax1}\PY{o}{.}\PY{n}{set\PYZus{}xlabel}\PY{p}{(}\PY{l+s+s1}{\PYZsq{}}\PY{l+s+s1}{Time}\PY{l+s+s1}{\PYZsq{}}\PY{p}{)}
\PY{n}{ax1}\PY{o}{.}\PY{n}{set\PYZus{}ylabel}\PY{p}{(}\PY{l+s+s1}{\PYZsq{}}\PY{l+s+s1}{Velocity}\PY{l+s+s1}{\PYZsq{}}\PY{p}{)}
\PY{n}{ax1}\PY{o}{.}\PY{n}{legend}\PY{p}{(}\PY{p}{)}
\PY{n}{ax1}\PY{o}{.}\PY{n}{grid}\PY{p}{(}\PY{p}{)}

\PY{n}{ax2} \PY{o}{=} \PY{n}{fig}\PY{o}{.}\PY{n}{add\PYZus{}subplot}\PY{p}{(}\PY{n}{gs}\PY{p}{[}\PY{p}{:}\PY{p}{,} \PY{l+m+mi}{1}\PY{p}{]}\PY{p}{)}
\PY{n}{ax2}\PY{o}{.}\PY{n}{plot}\PY{p}{(}\PY{n}{solution\PYZus{}short\PYZus{}theta}\PY{p}{,} \PY{n}{solution\PYZus{}short\PYZus{}omega}\PY{p}{)}
\PY{n}{ax2}\PY{o}{.}\PY{n}{set\PYZus{}title}\PY{p}{(}\PY{l+s+s1}{\PYZsq{}}\PY{l+s+s1}{Phase Space}\PY{l+s+s1}{\PYZsq{}}\PY{p}{)}
\PY{n}{ax2}\PY{o}{.}\PY{n}{scatter}\PY{p}{(}\PY{n}{solution\PYZus{}short\PYZus{}theta}\PY{p}{[}\PY{l+m+mi}{0}\PY{p}{]}\PY{p}{,} \PY{n}{solution\PYZus{}short\PYZus{}omega}\PY{p}{[}\PY{l+m+mi}{0}\PY{p}{]}\PY{p}{,} \PY{n}{color}\PY{o}{=}\PY{l+s+s1}{\PYZsq{}}\PY{l+s+s1}{C1}\PY{l+s+s1}{\PYZsq{}}\PY{p}{,} \PY{n}{marker}\PY{o}{=}\PY{l+s+s1}{\PYZsq{}}\PY{l+s+s1}{o}\PY{l+s+s1}{\PYZsq{}}\PY{p}{,} \PY{n}{label}\PY{o}{=}\PY{l+s+s1}{\PYZsq{}}\PY{l+s+s1}{Start}\PY{l+s+s1}{\PYZsq{}}\PY{p}{,} \PY{n}{s}\PY{o}{=}\PY{l+m+mi}{50}\PY{p}{)}
\PY{n}{ax2}\PY{o}{.}\PY{n}{scatter}\PY{p}{(}\PY{n}{solution\PYZus{}short\PYZus{}theta}\PY{p}{[}\PY{o}{\PYZhy{}}\PY{l+m+mi}{1}\PY{p}{]}\PY{p}{,} \PY{n}{solution\PYZus{}short\PYZus{}omega}\PY{p}{[}\PY{o}{\PYZhy{}}\PY{l+m+mi}{1}\PY{p}{]}\PY{p}{,} \PY{n}{color}\PY{o}{=}\PY{l+s+s1}{\PYZsq{}}\PY{l+s+s1}{C2}\PY{l+s+s1}{\PYZsq{}}\PY{p}{,} \PY{n}{marker}\PY{o}{=}\PY{l+s+s1}{\PYZsq{}}\PY{l+s+s1}{s}\PY{l+s+s1}{\PYZsq{}}\PY{p}{,} \PY{n}{label}\PY{o}{=}\PY{l+s+s1}{\PYZsq{}}\PY{l+s+s1}{End}\PY{l+s+s1}{\PYZsq{}}\PY{p}{,} \PY{n}{s}\PY{o}{=}\PY{l+m+mi}{50}\PY{p}{)}
\PY{n}{ax2}\PY{o}{.}\PY{n}{scatter}\PY{p}{(}\PY{n}{solution\PYZus{}short\PYZus{}theta}\PY{p}{[}\PY{n}{indices}\PY{p}{]}\PY{p}{,} \PY{n}{solution\PYZus{}short\PYZus{}omega}\PY{p}{[}\PY{n}{indices}\PY{p}{]}\PY{p}{,} \PY{n}{color}\PY{o}{=}\PY{l+s+s1}{\PYZsq{}}\PY{l+s+s1}{C3}\PY{l+s+s1}{\PYZsq{}}\PY{p}{,} \PY{n}{marker}\PY{o}{=}\PY{l+s+s1}{\PYZsq{}}\PY{l+s+s1}{x}\PY{l+s+s1}{\PYZsq{}}\PY{p}{,} \PY{n}{s}\PY{o}{=}\PY{l+m+mi}{50}\PY{p}{)}
\PY{n}{ax2}\PY{o}{.}\PY{n}{set\PYZus{}xlabel}\PY{p}{(}\PY{l+s+s1}{\PYZsq{}}\PY{l+s+s1}{Position}\PY{l+s+s1}{\PYZsq{}}\PY{p}{)}
\PY{n}{ax2}\PY{o}{.}\PY{n}{set\PYZus{}ylabel}\PY{p}{(}\PY{l+s+s1}{\PYZsq{}}\PY{l+s+s1}{Velocity}\PY{l+s+s1}{\PYZsq{}}\PY{p}{)}
\PY{n}{ax2}\PY{o}{.}\PY{n}{legend}\PY{p}{(}\PY{p}{)}
\PY{n}{ax2}\PY{o}{.}\PY{n}{grid}\PY{p}{(}\PY{p}{)}
\PY{n}{plt}\PY{o}{.}\PY{n}{tight\PYZus{}layout}\PY{p}{(}\PY{p}{)}
\end{Verbatim}
\end{tcolorbox}

    \begin{center}
    \adjustimage{max size={0.9\linewidth}{0.9\paperheight}}{10_notes_files/10_notes_7_0.png}
    \end{center}
    { \hspace*{\fill} \\}
    
    How does this view of the long term solution help us see the periodicity
of the motion? How could we check if the motion follows the driving
frequency?

\textbf{Your Answer Here}

    \subsubsection{A Period-1 Solution}\label{a-period-1-solution}

While the specific location of the marked points might vary a little
but, we can see that there's roughly a repeated location of those
points. In the case of position or velocity, the pink ex marks the spot
where the system returns to the same value. The phase space makes that
more clear because we can see that the system returns to the roughly the
same point in phase space. This is close to a period-1 solution. It
might be that we still have some transient behavior, or that our
integrator is not perfectly accurate, or that our scheme for selecting
the time points is not perfect, but we can see that the system is close
to periodic.

We can use this approach to investigate other chaotic systems and
explore their behavior over time.

    \subsection{Period Doubling with the Duffing
Oscillator}\label{period-doubling-with-the-duffing-oscillator}

The Duffing oscillator is a non-linear second-order differential
equation that describes the motion of a damped and driven oscillator
with a non-linear restoring force. Under certain conditions, it can
exhibit chaotic behavior, including period-doubling bifurcations. There
are many routes to chaos, but period-doubling is a classic example.

The model is given by:

\[\frac{d^2x}{dt^2} + \delta \frac{dx}{dt} + \alpha x + \beta x^3 = \gamma \cos(\omega t)\]

We can rewrite this as a system of first-order ODEs by introducing a new
variable for the velocity:

\[\dot{x} = v\]
\[\dot{v} = -\delta v - \alpha x - \beta x^3 + \gamma \cos(\omega t).\]

\subsubsection{Numerically Integrate the Duffing
Oscillator}\label{numerically-integrate-the-duffing-oscillator}

For our simulation we will observe the effect of the driving strength
\(\gamma\) on the behavior of the system. We start with the following
parameters:

\begin{longtable}[]{@{}ll@{}}
\toprule\noalign{}
Parameter & Value \\
\midrule\noalign{}
\endhead
\bottomrule\noalign{}
\endlastfoot
\(\delta\) & 0.2 \\
\(\alpha\) & -1 \\
\(\beta\) & 1 \\
\(\gamma\) & 0.3 \\
\(\omega\) & 1.2 \\
\end{longtable}

Choose the initial conditions and time span for the simulation:
\textbar{} Condition \textbar{} Value \textbar{} \textbar{} ---------
\textbar{} ----- \textbar{} \textbar{} Initial \(x\) \textbar{} 1.0
\textbar{} \textbar{} Initial \(v\) \textbar{} 0 \textbar{} \textbar{}
Time span \textbar{} 100 cycles of the driver \textbar{} \textbar{}
Evaluation points \textbar{} 10000 \textbar{}

Below we've provided some of the code, but not the definition of the
\texttt{duffing} function. You will need to write that function and fill
in the missing code to complete the simulation.

\textbf{Write the rest of the code necessary to simulate the Duffing
oscillator.}

\textbf{Plot the results of your simulation in the next cell.}

    \begin{tcolorbox}[breakable, size=fbox, boxrule=1pt, pad at break*=1mm,colback=cellbackground, colframe=cellborder]
\prompt{In}{incolor}{4}{\boxspacing}
\begin{Verbatim}[commandchars=\\\{\}]
\PY{k+kn}{import}\PY{+w}{ }\PY{n+nn}{numpy}\PY{+w}{ }\PY{k}{as}\PY{+w}{ }\PY{n+nn}{np}
\PY{k+kn}{import}\PY{+w}{ }\PY{n+nn}{matplotlib}\PY{n+nn}{.}\PY{n+nn}{pyplot}\PY{+w}{ }\PY{k}{as}\PY{+w}{ }\PY{n+nn}{plt}
\PY{k+kn}{from}\PY{+w}{ }\PY{n+nn}{matplotlib}\PY{+w}{ }\PY{k+kn}{import} \PY{n}{gridspec}
\PY{k+kn}{from}\PY{+w}{ }\PY{n+nn}{scipy}\PY{n+nn}{.}\PY{n+nn}{integrate}\PY{+w}{ }\PY{k+kn}{import} \PY{n}{solve\PYZus{}ivp}
\PY{n}{plt}\PY{o}{.}\PY{n}{style}\PY{o}{.}\PY{n}{use}\PY{p}{(}\PY{l+s+s1}{\PYZsq{}}\PY{l+s+s1}{seaborn\PYZhy{}v0\PYZus{}8\PYZhy{}colorblind}\PY{l+s+s1}{\PYZsq{}}\PY{p}{)}
\end{Verbatim}
\end{tcolorbox}

    \begin{tcolorbox}[breakable, size=fbox, boxrule=1pt, pad at break*=1mm,colback=cellbackground, colframe=cellborder]
\prompt{In}{incolor}{5}{\boxspacing}
\begin{Verbatim}[commandchars=\\\{\}]
\PY{c+c1}{\PYZsh{}\PYZsh{} def duffing(t, y, delta, alpha, beta, gamma, omega):}
    \PY{c+c1}{\PYZsh{}\PYZsh{} Your Code Here}

\PY{c+c1}{\PYZsh{} Parameters}
\PY{n}{delta} \PY{o}{=} \PY{l+m+mf}{0.3}
\PY{n}{alpha} \PY{o}{=} \PY{o}{\PYZhy{}}\PY{l+m+mf}{1.0}
\PY{n}{beta} \PY{o}{=} \PY{l+m+mf}{1.0}
\PY{n}{gamma} \PY{o}{=} \PY{l+m+mf}{0.2}
\PY{n}{omega} \PY{o}{=} \PY{l+m+mf}{1.2}

\PY{c+c1}{\PYZsh{} Time span and initial conditions}
\PY{n}{cycles} \PY{o}{=} \PY{l+m+mi}{100}
\PY{n}{tfinal} \PY{o}{=} \PY{l+m+mi}{2} \PY{o}{*} \PY{n}{np}\PY{o}{.}\PY{n}{pi} \PY{o}{*} \PY{n}{cycles} \PY{o}{/} \PY{n}{omega}
\PY{n}{t\PYZus{}span} \PY{o}{=} \PY{p}{(}\PY{l+m+mi}{0}\PY{p}{,} \PY{n}{tfinal}\PY{p}{)}
\PY{n}{t\PYZus{}eval} \PY{o}{=} \PY{n}{np}\PY{o}{.}\PY{n}{linspace}\PY{p}{(}\PY{n}{t\PYZus{}span}\PY{p}{[}\PY{l+m+mi}{0}\PY{p}{]}\PY{p}{,} \PY{n}{t\PYZus{}span}\PY{p}{[}\PY{l+m+mi}{1}\PY{p}{]}\PY{p}{,} \PY{l+m+mi}{10000}\PY{p}{)}
\PY{n}{y0} \PY{o}{=} \PY{p}{[}\PY{l+m+mf}{1.0}\PY{p}{,} \PY{l+m+mf}{0.0}\PY{p}{]}

\PY{c+c1}{\PYZsh{} Solve the ODEs using solve\PYZus{}ivp}
\PY{c+c1}{\PYZsh{}\PYZsh{} Your Code Here}
\end{Verbatim}
\end{tcolorbox}

    \subsubsection{Plotting the results}\label{plotting-the-results}

Once you have simulated the Duffing oscillator, you can plot the results
to visualize its behavior over time. Your plot should be similar to the
one above where you plot at least the position \(x\) over time, and then
a phase space plot of \(x\) vs.~\(v\). Make sure that you label your
axes and include a title for your plot; add starting and ending points
to your plot so you are sure about the direction of the motion.

This first set of plots should look similar to the one below.

\begin{figure}
\centering
\pandocbounded{\includegraphics[keepaspectratio,alt={Duffing}]{../images/10_notes_first_duffing.png}}
\caption{Duffing}
\end{figure}

    \begin{tcolorbox}[breakable, size=fbox, boxrule=1pt, pad at break*=1mm,colback=cellbackground, colframe=cellborder]
\prompt{In}{incolor}{6}{\boxspacing}
\begin{Verbatim}[commandchars=\\\{\}]
\PY{c+c1}{\PYZsh{}\PYZsh{} Plot the time series and phase space}
\PY{c+c1}{\PYZsh{}\PYZsh{} Your Code Here}
\end{Verbatim}
\end{tcolorbox}

    \subsubsection{Investigating
Periodicity}\label{investigating-periodicity}

Now that you have this solution, look at only the last few cycles of the
simulation. You can do this by plotting only the last 10 cycles of the
simulation. Include a mark at every cycle of the driver,
\(T = \frac{2\pi}{\omega}\). This will help you see if the system is
periodic or not.

    \begin{tcolorbox}[breakable, size=fbox, boxrule=1pt, pad at break*=1mm,colback=cellbackground, colframe=cellborder]
\prompt{In}{incolor}{7}{\boxspacing}
\begin{Verbatim}[commandchars=\\\{\}]
\PY{c+c1}{\PYZsh{}\PYZsh{} Plot the time series and phase space}
\PY{c+c1}{\PYZsh{}\PYZsh{} Your Code Here}
\end{Verbatim}
\end{tcolorbox}

    What did you find the periodicity of this particular set up to be? How
did you determine it?

\textbf{Your Answer Here}

    \subsubsection{Period Doubling}\label{period-doubling}

Let's use the same code, or write this code in a new cell, or write a
new function, to investigate the effect of increasing the driving
strength \(\gamma\) on the behavior of the system.

You will \textbf{run the code multiple times}, changing the value of
\(\gamma\) each time. You can do this by changing the value in the code,
or by writing a loop to run through a range of values. In either case,
you will need to plot the long term behavior of the system for each
value of \(\gamma\), including the phase space plot. Your plot should
also include a mark at every cycle of the driver,
\(T = \frac{2\pi}{\omega}\).

For choices of \(\gamma\), please use the following values: 0.2, 0.28,
0.29, 0.37, 0.50, 0.68, 0.74, 0.75,

    \begin{tcolorbox}[breakable, size=fbox, boxrule=1pt, pad at break*=1mm,colback=cellbackground, colframe=cellborder]
\prompt{In}{incolor}{8}{\boxspacing}
\begin{Verbatim}[commandchars=\\\{\}]
\PY{c+c1}{\PYZsh{}\PYZsh{} Your Code Here}
\end{Verbatim}
\end{tcolorbox}

    \subsection{Sensitivity to Initial Conditions with the Lorenz
Attractor}\label{sensitivity-to-initial-conditions-with-the-lorenz-attractor}

The Lorenz attractor is a system of ordinary differential equations that
model atmospheric convection. It is a classic example of a chaotic
system that exhibits sensitive dependence on initial conditions. It also
demonstrates the phenomenon of a
\href{https://en.wikipedia.org/wiki/Strange_attractor}{strange
attractor}. A strange attractor is a fractal structure in phase space
that the system approaches asymptotically over time. For the Lorenz
attractor, the strange attractor is a set of points in phase space that
the system approaches as time goes to infinity - the butterfly shape
that you might be familiar with.

In this activity, we will simulate the Lorenz attractor using
\texttt{solve\_ivp} and explore how solutions diverge from each other
based on small differences in initial conditions.

\subsubsection{Mathematical Model}\label{mathematical-model}

The Lorenz model is given by:

\[\frac{dx}{dt} = \sigma (y - x)\] \[\frac{dy}{dt} = x(\rho - z) - y\]
\[\frac{dz}{dt} = xy - \beta z\]

Where \(\sigma\), \(\rho\), and \(\beta\) are system parameters. The
canonical values are \(\sigma = 10\), \(\rho = 28\), and
\(\beta = \frac{8}{3}\).

\subsubsection{Numerically Integrate the Lorenz
Attractor}\label{numerically-integrate-the-lorenz-attractor}

In the cells below, we scaffold some of the code to simulate the Lorenz
attractor. You will need to fill in the missing pieces. Once you plot
the solution, you should be able to produce time series, and phase space
plots of the Lorenz attractor. \textbf{Note that the phase space for the
Lorenz attractor is 3D (x,y,z), so you will need to use a 3D plotting
function or plot projections.}

For the parameters, we will use the canonical values of \(\sigma = 10\),
\(\rho = 28\), and \(\beta = \frac{8}{3}\). Choose initial conditions of
\(x=1\), \(y=1\), and \(z=1\) and simulate for 50 time units. If you do,
your solution will look like the one below.

\begin{figure}
\centering
\pandocbounded{\includegraphics[keepaspectratio,alt={Lorenz Time Series}]{../images/10_notes_lorenz-1.png}}
\caption{Lorenz Time Series}
\end{figure}

\begin{figure}
\centering
\pandocbounded{\includegraphics[keepaspectratio,alt={Lorenz Phase Space}]{../images/10_notes_lorenz-2.png}}
\caption{Lorenz Phase Space}
\end{figure}

    \begin{tcolorbox}[breakable, size=fbox, boxrule=1pt, pad at break*=1mm,colback=cellbackground, colframe=cellborder]
\prompt{In}{incolor}{9}{\boxspacing}
\begin{Verbatim}[commandchars=\\\{\}]
\PY{k+kn}{import}\PY{+w}{ }\PY{n+nn}{numpy}\PY{+w}{ }\PY{k}{as}\PY{+w}{ }\PY{n+nn}{np}
\PY{k+kn}{import}\PY{+w}{ }\PY{n+nn}{matplotlib}\PY{n+nn}{.}\PY{n+nn}{pyplot}\PY{+w}{ }\PY{k}{as}\PY{+w}{ }\PY{n+nn}{plt}
\PY{k+kn}{from}\PY{+w}{ }\PY{n+nn}{matplotlib}\PY{+w}{ }\PY{k+kn}{import} \PY{n}{gridspec}
\PY{k+kn}{from}\PY{+w}{ }\PY{n+nn}{scipy}\PY{n+nn}{.}\PY{n+nn}{integrate}\PY{+w}{ }\PY{k+kn}{import} \PY{n}{solve\PYZus{}ivp}
\PY{n}{plt}\PY{o}{.}\PY{n}{style}\PY{o}{.}\PY{n}{use}\PY{p}{(}\PY{l+s+s1}{\PYZsq{}}\PY{l+s+s1}{seaborn\PYZhy{}v0\PYZus{}8\PYZhy{}colorblind}\PY{l+s+s1}{\PYZsq{}}\PY{p}{)}
\end{Verbatim}
\end{tcolorbox}

    \begin{tcolorbox}[breakable, size=fbox, boxrule=1pt, pad at break*=1mm,colback=cellbackground, colframe=cellborder]
\prompt{In}{incolor}{10}{\boxspacing}
\begin{Verbatim}[commandchars=\\\{\}]
\PY{c+c1}{\PYZsh{}\PYZsh{} def lorenz(t, y, sigma, beta, rho):}
    \PY{c+c1}{\PYZsh{}\PYZsh{} Your Code Here}

\PY{c+c1}{\PYZsh{} Parameters for the Lorenz system}
\PY{n}{sigma} \PY{o}{=} \PY{l+m+mf}{10.0}
\PY{n}{beta} \PY{o}{=} \PY{l+m+mi}{8}\PY{o}{/}\PY{l+m+mi}{3}
\PY{n}{rho} \PY{o}{=} \PY{l+m+mf}{28.0}

\PY{c+c1}{\PYZsh{} Time span and initial conditions}
\PY{n}{t\PYZus{}span} \PY{o}{=} \PY{p}{(}\PY{l+m+mi}{0}\PY{p}{,} \PY{l+m+mi}{50}\PY{p}{)}
\PY{n}{t\PYZus{}eval} \PY{o}{=} \PY{n}{np}\PY{o}{.}\PY{n}{linspace}\PY{p}{(}\PY{n}{t\PYZus{}span}\PY{p}{[}\PY{l+m+mi}{0}\PY{p}{]}\PY{p}{,} \PY{n}{t\PYZus{}span}\PY{p}{[}\PY{l+m+mi}{1}\PY{p}{]}\PY{p}{,} \PY{l+m+mi}{10000}\PY{p}{)}
\PY{n}{y0} \PY{o}{=} \PY{p}{[}\PY{l+m+mf}{1.0}\PY{p}{,} \PY{l+m+mf}{1.0}\PY{p}{,} \PY{l+m+mf}{1.0}\PY{p}{]}

\PY{c+c1}{\PYZsh{} Solve the differential equations using solve\PYZus{}ivp}
\PY{c+c1}{\PYZsh{}\PYZsh{} Your Code Here}

\PY{c+c1}{\PYZsh{} Plotting your results}
\PY{c+c1}{\PYZsh{}\PYZsh{} Your Code Here}
\end{Verbatim}
\end{tcolorbox}

    \subsubsection{Exploring Sensitivity to Initial
Conditions}\label{exploring-sensitivity-to-initial-conditions}

Noe that you have a solution to the Lorenz attractor, you can observe
how trajectories that start from slightly different initial conditions
diverge over time. This is a hallmark of chaotic systems, where small
differences in initial conditions can lead to vastly different outcomes.
We can visualize this as a bundle of trajectories that start from nearby
points in phase space and then diverge over time. For the Lorenz
attractor, this is often visualized as a butterfly-shaped structure in
phase space.

For this activity, we will simulate the Lorenz attractor for two sets of
initial conditions that are close together. We will then plot the
trajectories in phase space to observe how they diverge over time.

\textbf{Modify the code you wrote above to simulate the Lorenz attractor
for two sets of initial conditions that are close together. Plot the
trajectories in phase space to observe how they diverge over time. Do
this as both a time series and in phase space.}

    \begin{tcolorbox}[breakable, size=fbox, boxrule=1pt, pad at break*=1mm,colback=cellbackground, colframe=cellborder]
\prompt{In}{incolor}{11}{\boxspacing}
\begin{Verbatim}[commandchars=\\\{\}]
\PY{c+c1}{\PYZsh{}\PYZsh{} Your Code Here}
\end{Verbatim}
\end{tcolorbox}

    \subsubsection{The Strange Attractor}\label{the-strange-attractor}

The two trajectories that you plotted above should look like they are
diverging over time. This is very common in chaotic systems. However,
when you plot the trajectories in the phase space, they seem to occupy a
bounded region of space - this is especially true as the trajectories
evolve.

This bounded region is called a strange attractor. A strange attractor
is a set of points in phase space that the system approaches
asymptotically over time. For the Lorenz attractor, the strange
attractor is a fractal structure that the system approaches as time goes
to infinity.

Let's plot a ton of trajectories starting from different initial
conditions to see the strange attractor. We can do this by looping over
a range of initial conditions and plotting each trajectory in phase
space.

That will get very messy, so instead, let's integrate 100-1000
trajectories that are near the original trajectory. We will plot only
the original and final location of each trajectory. This will give us a
sense of the strange attractor without plotting every single trajectory.

    \begin{tcolorbox}[breakable, size=fbox, boxrule=1pt, pad at break*=1mm,colback=cellbackground, colframe=cellborder]
\prompt{In}{incolor}{12}{\boxspacing}
\begin{Verbatim}[commandchars=\\\{\}]
\PY{c+c1}{\PYZsh{}\PYZsh{} Your Code Here}
\end{Verbatim}
\end{tcolorbox}


    % Add a bibliography block to the postdoc
    
    
    
\end{document}

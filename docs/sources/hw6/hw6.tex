\documentclass[11pt]{article}

    \usepackage[breakable]{tcolorbox}
    \usepackage{parskip} % Stop auto-indenting (to mimic markdown behaviour)
    

    % Basic figure setup, for now with no caption control since it's done
    % automatically by Pandoc (which extracts ![](path) syntax from Markdown).
    \usepackage{graphicx}
    % Keep aspect ratio if custom image width or height is specified
    \setkeys{Gin}{keepaspectratio}
    % Maintain compatibility with old templates. Remove in nbconvert 6.0
    \let\Oldincludegraphics\includegraphics
    % Ensure that by default, figures have no caption (until we provide a
    % proper Figure object with a Caption API and a way to capture that
    % in the conversion process - todo).
    \usepackage{caption}
    \DeclareCaptionFormat{nocaption}{}
    \captionsetup{format=nocaption,aboveskip=0pt,belowskip=0pt}

    \usepackage{float}
    \floatplacement{figure}{H} % forces figures to be placed at the correct location
    \usepackage{xcolor} % Allow colors to be defined
    \usepackage{enumerate} % Needed for markdown enumerations to work
    \usepackage{geometry} % Used to adjust the document margins
    \usepackage{amsmath} % Equations
    \usepackage{amssymb} % Equations
    \usepackage{textcomp} % defines textquotesingle
    % Hack from http://tex.stackexchange.com/a/47451/13684:
    \AtBeginDocument{%
        \def\PYZsq{\textquotesingle}% Upright quotes in Pygmentized code
    }
    \usepackage{upquote} % Upright quotes for verbatim code
    \usepackage{eurosym} % defines \euro

    \usepackage{iftex}
    \ifPDFTeX
        \usepackage[T1]{fontenc}
        \IfFileExists{alphabeta.sty}{
              \usepackage{alphabeta}
          }{
              \usepackage[mathletters]{ucs}
              \usepackage[utf8x]{inputenc}
          }
    \else
        \usepackage{fontspec}
        \usepackage{unicode-math}
    \fi

    \usepackage{fancyvrb} % verbatim replacement that allows latex
    \usepackage{grffile} % extends the file name processing of package graphics
                         % to support a larger range
    \makeatletter % fix for old versions of grffile with XeLaTeX
    \@ifpackagelater{grffile}{2019/11/01}
    {
      % Do nothing on new versions
    }
    {
      \def\Gread@@xetex#1{%
        \IfFileExists{"\Gin@base".bb}%
        {\Gread@eps{\Gin@base.bb}}%
        {\Gread@@xetex@aux#1}%
      }
    }
    \makeatother
    \usepackage[Export]{adjustbox} % Used to constrain images to a maximum size
    \adjustboxset{max size={0.9\linewidth}{0.9\paperheight}}

    % The hyperref package gives us a pdf with properly built
    % internal navigation ('pdf bookmarks' for the table of contents,
    % internal cross-reference links, web links for URLs, etc.)
    \usepackage{hyperref}
    % The default LaTeX title has an obnoxious amount of whitespace. By default,
    % titling removes some of it. It also provides customization options.
    \usepackage{titling}
    \usepackage{longtable} % longtable support required by pandoc >1.10
    \usepackage{booktabs}  % table support for pandoc > 1.12.2
    \usepackage{array}     % table support for pandoc >= 2.11.3
    \usepackage{calc}      % table minipage width calculation for pandoc >= 2.11.1
    \usepackage[inline]{enumitem} % IRkernel/repr support (it uses the enumerate* environment)
    \usepackage[normalem]{ulem} % ulem is needed to support strikethroughs (\sout)
                                % normalem makes italics be italics, not underlines
    \usepackage{soul}      % strikethrough (\st) support for pandoc >= 3.0.0
    \usepackage{mathrsfs}
    

    
    % Colors for the hyperref package
    \definecolor{urlcolor}{rgb}{0,.145,.698}
    \definecolor{linkcolor}{rgb}{.71,0.21,0.01}
    \definecolor{citecolor}{rgb}{.12,.54,.11}

    % ANSI colors
    \definecolor{ansi-black}{HTML}{3E424D}
    \definecolor{ansi-black-intense}{HTML}{282C36}
    \definecolor{ansi-red}{HTML}{E75C58}
    \definecolor{ansi-red-intense}{HTML}{B22B31}
    \definecolor{ansi-green}{HTML}{00A250}
    \definecolor{ansi-green-intense}{HTML}{007427}
    \definecolor{ansi-yellow}{HTML}{DDB62B}
    \definecolor{ansi-yellow-intense}{HTML}{B27D12}
    \definecolor{ansi-blue}{HTML}{208FFB}
    \definecolor{ansi-blue-intense}{HTML}{0065CA}
    \definecolor{ansi-magenta}{HTML}{D160C4}
    \definecolor{ansi-magenta-intense}{HTML}{A03196}
    \definecolor{ansi-cyan}{HTML}{60C6C8}
    \definecolor{ansi-cyan-intense}{HTML}{258F8F}
    \definecolor{ansi-white}{HTML}{C5C1B4}
    \definecolor{ansi-white-intense}{HTML}{A1A6B2}
    \definecolor{ansi-default-inverse-fg}{HTML}{FFFFFF}
    \definecolor{ansi-default-inverse-bg}{HTML}{000000}

    % common color for the border for error outputs.
    \definecolor{outerrorbackground}{HTML}{FFDFDF}

    % commands and environments needed by pandoc snippets
    % extracted from the output of `pandoc -s`
    \providecommand{\tightlist}{%
      \setlength{\itemsep}{0pt}\setlength{\parskip}{0pt}}
    \DefineVerbatimEnvironment{Highlighting}{Verbatim}{commandchars=\\\{\}}
    % Add ',fontsize=\small' for more characters per line
    \newenvironment{Shaded}{}{}
    \newcommand{\KeywordTok}[1]{\textcolor[rgb]{0.00,0.44,0.13}{\textbf{{#1}}}}
    \newcommand{\DataTypeTok}[1]{\textcolor[rgb]{0.56,0.13,0.00}{{#1}}}
    \newcommand{\DecValTok}[1]{\textcolor[rgb]{0.25,0.63,0.44}{{#1}}}
    \newcommand{\BaseNTok}[1]{\textcolor[rgb]{0.25,0.63,0.44}{{#1}}}
    \newcommand{\FloatTok}[1]{\textcolor[rgb]{0.25,0.63,0.44}{{#1}}}
    \newcommand{\CharTok}[1]{\textcolor[rgb]{0.25,0.44,0.63}{{#1}}}
    \newcommand{\StringTok}[1]{\textcolor[rgb]{0.25,0.44,0.63}{{#1}}}
    \newcommand{\CommentTok}[1]{\textcolor[rgb]{0.38,0.63,0.69}{\textit{{#1}}}}
    \newcommand{\OtherTok}[1]{\textcolor[rgb]{0.00,0.44,0.13}{{#1}}}
    \newcommand{\AlertTok}[1]{\textcolor[rgb]{1.00,0.00,0.00}{\textbf{{#1}}}}
    \newcommand{\FunctionTok}[1]{\textcolor[rgb]{0.02,0.16,0.49}{{#1}}}
    \newcommand{\RegionMarkerTok}[1]{{#1}}
    \newcommand{\ErrorTok}[1]{\textcolor[rgb]{1.00,0.00,0.00}{\textbf{{#1}}}}
    \newcommand{\NormalTok}[1]{{#1}}

    % Additional commands for more recent versions of Pandoc
    \newcommand{\ConstantTok}[1]{\textcolor[rgb]{0.53,0.00,0.00}{{#1}}}
    \newcommand{\SpecialCharTok}[1]{\textcolor[rgb]{0.25,0.44,0.63}{{#1}}}
    \newcommand{\VerbatimStringTok}[1]{\textcolor[rgb]{0.25,0.44,0.63}{{#1}}}
    \newcommand{\SpecialStringTok}[1]{\textcolor[rgb]{0.73,0.40,0.53}{{#1}}}
    \newcommand{\ImportTok}[1]{{#1}}
    \newcommand{\DocumentationTok}[1]{\textcolor[rgb]{0.73,0.13,0.13}{\textit{{#1}}}}
    \newcommand{\AnnotationTok}[1]{\textcolor[rgb]{0.38,0.63,0.69}{\textbf{\textit{{#1}}}}}
    \newcommand{\CommentVarTok}[1]{\textcolor[rgb]{0.38,0.63,0.69}{\textbf{\textit{{#1}}}}}
    \newcommand{\VariableTok}[1]{\textcolor[rgb]{0.10,0.09,0.49}{{#1}}}
    \newcommand{\ControlFlowTok}[1]{\textcolor[rgb]{0.00,0.44,0.13}{\textbf{{#1}}}}
    \newcommand{\OperatorTok}[1]{\textcolor[rgb]{0.40,0.40,0.40}{{#1}}}
    \newcommand{\BuiltInTok}[1]{{#1}}
    \newcommand{\ExtensionTok}[1]{{#1}}
    \newcommand{\PreprocessorTok}[1]{\textcolor[rgb]{0.74,0.48,0.00}{{#1}}}
    \newcommand{\AttributeTok}[1]{\textcolor[rgb]{0.49,0.56,0.16}{{#1}}}
    \newcommand{\InformationTok}[1]{\textcolor[rgb]{0.38,0.63,0.69}{\textbf{\textit{{#1}}}}}
    \newcommand{\WarningTok}[1]{\textcolor[rgb]{0.38,0.63,0.69}{\textbf{\textit{{#1}}}}}
    \makeatletter
    \newsavebox\pandoc@box
    \newcommand*\pandocbounded[1]{%
      \sbox\pandoc@box{#1}%
      % scaling factors for width and height
      \Gscale@div\@tempa\textheight{\dimexpr\ht\pandoc@box+\dp\pandoc@box\relax}%
      \Gscale@div\@tempb\linewidth{\wd\pandoc@box}%
      % select the smaller of both
      \ifdim\@tempb\p@<\@tempa\p@
        \let\@tempa\@tempb
      \fi
      % scaling accordingly (\@tempa < 1)
      \ifdim\@tempa\p@<\p@
        \scalebox{\@tempa}{\usebox\pandoc@box}%
      % scaling not needed, use as it is
      \else
        \usebox{\pandoc@box}%
      \fi
    }
    \makeatother

    % Define a nice break command that doesn't care if a line doesn't already
    % exist.
    \def\br{\hspace*{\fill} \\* }
    % Math Jax compatibility definitions
    \def\gt{>}
    \def\lt{<}
    \let\Oldtex\TeX
    \let\Oldlatex\LaTeX
    \renewcommand{\TeX}{\textrm{\Oldtex}}
    \renewcommand{\LaTeX}{\textrm{\Oldlatex}}
    % Document parameters
    % Document title
    \title{hw6}
    
    
    
    
    
    
    
% Pygments definitions
\makeatletter
\def\PY@reset{\let\PY@it=\relax \let\PY@bf=\relax%
    \let\PY@ul=\relax \let\PY@tc=\relax%
    \let\PY@bc=\relax \let\PY@ff=\relax}
\def\PY@tok#1{\csname PY@tok@#1\endcsname}
\def\PY@toks#1+{\ifx\relax#1\empty\else%
    \PY@tok{#1}\expandafter\PY@toks\fi}
\def\PY@do#1{\PY@bc{\PY@tc{\PY@ul{%
    \PY@it{\PY@bf{\PY@ff{#1}}}}}}}
\def\PY#1#2{\PY@reset\PY@toks#1+\relax+\PY@do{#2}}

\@namedef{PY@tok@w}{\def\PY@tc##1{\textcolor[rgb]{0.73,0.73,0.73}{##1}}}
\@namedef{PY@tok@c}{\let\PY@it=\textit\def\PY@tc##1{\textcolor[rgb]{0.24,0.48,0.48}{##1}}}
\@namedef{PY@tok@cp}{\def\PY@tc##1{\textcolor[rgb]{0.61,0.40,0.00}{##1}}}
\@namedef{PY@tok@k}{\let\PY@bf=\textbf\def\PY@tc##1{\textcolor[rgb]{0.00,0.50,0.00}{##1}}}
\@namedef{PY@tok@kp}{\def\PY@tc##1{\textcolor[rgb]{0.00,0.50,0.00}{##1}}}
\@namedef{PY@tok@kt}{\def\PY@tc##1{\textcolor[rgb]{0.69,0.00,0.25}{##1}}}
\@namedef{PY@tok@o}{\def\PY@tc##1{\textcolor[rgb]{0.40,0.40,0.40}{##1}}}
\@namedef{PY@tok@ow}{\let\PY@bf=\textbf\def\PY@tc##1{\textcolor[rgb]{0.67,0.13,1.00}{##1}}}
\@namedef{PY@tok@nb}{\def\PY@tc##1{\textcolor[rgb]{0.00,0.50,0.00}{##1}}}
\@namedef{PY@tok@nf}{\def\PY@tc##1{\textcolor[rgb]{0.00,0.00,1.00}{##1}}}
\@namedef{PY@tok@nc}{\let\PY@bf=\textbf\def\PY@tc##1{\textcolor[rgb]{0.00,0.00,1.00}{##1}}}
\@namedef{PY@tok@nn}{\let\PY@bf=\textbf\def\PY@tc##1{\textcolor[rgb]{0.00,0.00,1.00}{##1}}}
\@namedef{PY@tok@ne}{\let\PY@bf=\textbf\def\PY@tc##1{\textcolor[rgb]{0.80,0.25,0.22}{##1}}}
\@namedef{PY@tok@nv}{\def\PY@tc##1{\textcolor[rgb]{0.10,0.09,0.49}{##1}}}
\@namedef{PY@tok@no}{\def\PY@tc##1{\textcolor[rgb]{0.53,0.00,0.00}{##1}}}
\@namedef{PY@tok@nl}{\def\PY@tc##1{\textcolor[rgb]{0.46,0.46,0.00}{##1}}}
\@namedef{PY@tok@ni}{\let\PY@bf=\textbf\def\PY@tc##1{\textcolor[rgb]{0.44,0.44,0.44}{##1}}}
\@namedef{PY@tok@na}{\def\PY@tc##1{\textcolor[rgb]{0.41,0.47,0.13}{##1}}}
\@namedef{PY@tok@nt}{\let\PY@bf=\textbf\def\PY@tc##1{\textcolor[rgb]{0.00,0.50,0.00}{##1}}}
\@namedef{PY@tok@nd}{\def\PY@tc##1{\textcolor[rgb]{0.67,0.13,1.00}{##1}}}
\@namedef{PY@tok@s}{\def\PY@tc##1{\textcolor[rgb]{0.73,0.13,0.13}{##1}}}
\@namedef{PY@tok@sd}{\let\PY@it=\textit\def\PY@tc##1{\textcolor[rgb]{0.73,0.13,0.13}{##1}}}
\@namedef{PY@tok@si}{\let\PY@bf=\textbf\def\PY@tc##1{\textcolor[rgb]{0.64,0.35,0.47}{##1}}}
\@namedef{PY@tok@se}{\let\PY@bf=\textbf\def\PY@tc##1{\textcolor[rgb]{0.67,0.36,0.12}{##1}}}
\@namedef{PY@tok@sr}{\def\PY@tc##1{\textcolor[rgb]{0.64,0.35,0.47}{##1}}}
\@namedef{PY@tok@ss}{\def\PY@tc##1{\textcolor[rgb]{0.10,0.09,0.49}{##1}}}
\@namedef{PY@tok@sx}{\def\PY@tc##1{\textcolor[rgb]{0.00,0.50,0.00}{##1}}}
\@namedef{PY@tok@m}{\def\PY@tc##1{\textcolor[rgb]{0.40,0.40,0.40}{##1}}}
\@namedef{PY@tok@gh}{\let\PY@bf=\textbf\def\PY@tc##1{\textcolor[rgb]{0.00,0.00,0.50}{##1}}}
\@namedef{PY@tok@gu}{\let\PY@bf=\textbf\def\PY@tc##1{\textcolor[rgb]{0.50,0.00,0.50}{##1}}}
\@namedef{PY@tok@gd}{\def\PY@tc##1{\textcolor[rgb]{0.63,0.00,0.00}{##1}}}
\@namedef{PY@tok@gi}{\def\PY@tc##1{\textcolor[rgb]{0.00,0.52,0.00}{##1}}}
\@namedef{PY@tok@gr}{\def\PY@tc##1{\textcolor[rgb]{0.89,0.00,0.00}{##1}}}
\@namedef{PY@tok@ge}{\let\PY@it=\textit}
\@namedef{PY@tok@gs}{\let\PY@bf=\textbf}
\@namedef{PY@tok@ges}{\let\PY@bf=\textbf\let\PY@it=\textit}
\@namedef{PY@tok@gp}{\let\PY@bf=\textbf\def\PY@tc##1{\textcolor[rgb]{0.00,0.00,0.50}{##1}}}
\@namedef{PY@tok@go}{\def\PY@tc##1{\textcolor[rgb]{0.44,0.44,0.44}{##1}}}
\@namedef{PY@tok@gt}{\def\PY@tc##1{\textcolor[rgb]{0.00,0.27,0.87}{##1}}}
\@namedef{PY@tok@err}{\def\PY@bc##1{{\setlength{\fboxsep}{\string -\fboxrule}\fcolorbox[rgb]{1.00,0.00,0.00}{1,1,1}{\strut ##1}}}}
\@namedef{PY@tok@kc}{\let\PY@bf=\textbf\def\PY@tc##1{\textcolor[rgb]{0.00,0.50,0.00}{##1}}}
\@namedef{PY@tok@kd}{\let\PY@bf=\textbf\def\PY@tc##1{\textcolor[rgb]{0.00,0.50,0.00}{##1}}}
\@namedef{PY@tok@kn}{\let\PY@bf=\textbf\def\PY@tc##1{\textcolor[rgb]{0.00,0.50,0.00}{##1}}}
\@namedef{PY@tok@kr}{\let\PY@bf=\textbf\def\PY@tc##1{\textcolor[rgb]{0.00,0.50,0.00}{##1}}}
\@namedef{PY@tok@bp}{\def\PY@tc##1{\textcolor[rgb]{0.00,0.50,0.00}{##1}}}
\@namedef{PY@tok@fm}{\def\PY@tc##1{\textcolor[rgb]{0.00,0.00,1.00}{##1}}}
\@namedef{PY@tok@vc}{\def\PY@tc##1{\textcolor[rgb]{0.10,0.09,0.49}{##1}}}
\@namedef{PY@tok@vg}{\def\PY@tc##1{\textcolor[rgb]{0.10,0.09,0.49}{##1}}}
\@namedef{PY@tok@vi}{\def\PY@tc##1{\textcolor[rgb]{0.10,0.09,0.49}{##1}}}
\@namedef{PY@tok@vm}{\def\PY@tc##1{\textcolor[rgb]{0.10,0.09,0.49}{##1}}}
\@namedef{PY@tok@sa}{\def\PY@tc##1{\textcolor[rgb]{0.73,0.13,0.13}{##1}}}
\@namedef{PY@tok@sb}{\def\PY@tc##1{\textcolor[rgb]{0.73,0.13,0.13}{##1}}}
\@namedef{PY@tok@sc}{\def\PY@tc##1{\textcolor[rgb]{0.73,0.13,0.13}{##1}}}
\@namedef{PY@tok@dl}{\def\PY@tc##1{\textcolor[rgb]{0.73,0.13,0.13}{##1}}}
\@namedef{PY@tok@s2}{\def\PY@tc##1{\textcolor[rgb]{0.73,0.13,0.13}{##1}}}
\@namedef{PY@tok@sh}{\def\PY@tc##1{\textcolor[rgb]{0.73,0.13,0.13}{##1}}}
\@namedef{PY@tok@s1}{\def\PY@tc##1{\textcolor[rgb]{0.73,0.13,0.13}{##1}}}
\@namedef{PY@tok@mb}{\def\PY@tc##1{\textcolor[rgb]{0.40,0.40,0.40}{##1}}}
\@namedef{PY@tok@mf}{\def\PY@tc##1{\textcolor[rgb]{0.40,0.40,0.40}{##1}}}
\@namedef{PY@tok@mh}{\def\PY@tc##1{\textcolor[rgb]{0.40,0.40,0.40}{##1}}}
\@namedef{PY@tok@mi}{\def\PY@tc##1{\textcolor[rgb]{0.40,0.40,0.40}{##1}}}
\@namedef{PY@tok@il}{\def\PY@tc##1{\textcolor[rgb]{0.40,0.40,0.40}{##1}}}
\@namedef{PY@tok@mo}{\def\PY@tc##1{\textcolor[rgb]{0.40,0.40,0.40}{##1}}}
\@namedef{PY@tok@ch}{\let\PY@it=\textit\def\PY@tc##1{\textcolor[rgb]{0.24,0.48,0.48}{##1}}}
\@namedef{PY@tok@cm}{\let\PY@it=\textit\def\PY@tc##1{\textcolor[rgb]{0.24,0.48,0.48}{##1}}}
\@namedef{PY@tok@cpf}{\let\PY@it=\textit\def\PY@tc##1{\textcolor[rgb]{0.24,0.48,0.48}{##1}}}
\@namedef{PY@tok@c1}{\let\PY@it=\textit\def\PY@tc##1{\textcolor[rgb]{0.24,0.48,0.48}{##1}}}
\@namedef{PY@tok@cs}{\let\PY@it=\textit\def\PY@tc##1{\textcolor[rgb]{0.24,0.48,0.48}{##1}}}

\def\PYZbs{\char`\\}
\def\PYZus{\char`\_}
\def\PYZob{\char`\{}
\def\PYZcb{\char`\}}
\def\PYZca{\char`\^}
\def\PYZam{\char`\&}
\def\PYZlt{\char`\<}
\def\PYZgt{\char`\>}
\def\PYZsh{\char`\#}
\def\PYZpc{\char`\%}
\def\PYZdl{\char`\$}
\def\PYZhy{\char`\-}
\def\PYZsq{\char`\'}
\def\PYZdq{\char`\"}
\def\PYZti{\char`\~}
% for compatibility with earlier versions
\def\PYZat{@}
\def\PYZlb{[}
\def\PYZrb{]}
\makeatother


    % For linebreaks inside Verbatim environment from package fancyvrb.
    \makeatletter
        \newbox\Wrappedcontinuationbox
        \newbox\Wrappedvisiblespacebox
        \newcommand*\Wrappedvisiblespace {\textcolor{red}{\textvisiblespace}}
        \newcommand*\Wrappedcontinuationsymbol {\textcolor{red}{\llap{\tiny$\m@th\hookrightarrow$}}}
        \newcommand*\Wrappedcontinuationindent {3ex }
        \newcommand*\Wrappedafterbreak {\kern\Wrappedcontinuationindent\copy\Wrappedcontinuationbox}
        % Take advantage of the already applied Pygments mark-up to insert
        % potential linebreaks for TeX processing.
        %        {, <, #, %, $, ' and ": go to next line.
        %        _, }, ^, &, >, - and ~: stay at end of broken line.
        % Use of \textquotesingle for straight quote.
        \newcommand*\Wrappedbreaksatspecials {%
            \def\PYGZus{\discretionary{\char`\_}{\Wrappedafterbreak}{\char`\_}}%
            \def\PYGZob{\discretionary{}{\Wrappedafterbreak\char`\{}{\char`\{}}%
            \def\PYGZcb{\discretionary{\char`\}}{\Wrappedafterbreak}{\char`\}}}%
            \def\PYGZca{\discretionary{\char`\^}{\Wrappedafterbreak}{\char`\^}}%
            \def\PYGZam{\discretionary{\char`\&}{\Wrappedafterbreak}{\char`\&}}%
            \def\PYGZlt{\discretionary{}{\Wrappedafterbreak\char`\<}{\char`\<}}%
            \def\PYGZgt{\discretionary{\char`\>}{\Wrappedafterbreak}{\char`\>}}%
            \def\PYGZsh{\discretionary{}{\Wrappedafterbreak\char`\#}{\char`\#}}%
            \def\PYGZpc{\discretionary{}{\Wrappedafterbreak\char`\%}{\char`\%}}%
            \def\PYGZdl{\discretionary{}{\Wrappedafterbreak\char`\$}{\char`\$}}%
            \def\PYGZhy{\discretionary{\char`\-}{\Wrappedafterbreak}{\char`\-}}%
            \def\PYGZsq{\discretionary{}{\Wrappedafterbreak\textquotesingle}{\textquotesingle}}%
            \def\PYGZdq{\discretionary{}{\Wrappedafterbreak\char`\"}{\char`\"}}%
            \def\PYGZti{\discretionary{\char`\~}{\Wrappedafterbreak}{\char`\~}}%
        }
        % Some characters . , ; ? ! / are not pygmentized.
        % This macro makes them "active" and they will insert potential linebreaks
        \newcommand*\Wrappedbreaksatpunct {%
            \lccode`\~`\.\lowercase{\def~}{\discretionary{\hbox{\char`\.}}{\Wrappedafterbreak}{\hbox{\char`\.}}}%
            \lccode`\~`\,\lowercase{\def~}{\discretionary{\hbox{\char`\,}}{\Wrappedafterbreak}{\hbox{\char`\,}}}%
            \lccode`\~`\;\lowercase{\def~}{\discretionary{\hbox{\char`\;}}{\Wrappedafterbreak}{\hbox{\char`\;}}}%
            \lccode`\~`\:\lowercase{\def~}{\discretionary{\hbox{\char`\:}}{\Wrappedafterbreak}{\hbox{\char`\:}}}%
            \lccode`\~`\?\lowercase{\def~}{\discretionary{\hbox{\char`\?}}{\Wrappedafterbreak}{\hbox{\char`\?}}}%
            \lccode`\~`\!\lowercase{\def~}{\discretionary{\hbox{\char`\!}}{\Wrappedafterbreak}{\hbox{\char`\!}}}%
            \lccode`\~`\/\lowercase{\def~}{\discretionary{\hbox{\char`\/}}{\Wrappedafterbreak}{\hbox{\char`\/}}}%
            \catcode`\.\active
            \catcode`\,\active
            \catcode`\;\active
            \catcode`\:\active
            \catcode`\?\active
            \catcode`\!\active
            \catcode`\/\active
            \lccode`\~`\~
        }
    \makeatother

    \let\OriginalVerbatim=\Verbatim
    \makeatletter
    \renewcommand{\Verbatim}[1][1]{%
        %\parskip\z@skip
        \sbox\Wrappedcontinuationbox {\Wrappedcontinuationsymbol}%
        \sbox\Wrappedvisiblespacebox {\FV@SetupFont\Wrappedvisiblespace}%
        \def\FancyVerbFormatLine ##1{\hsize\linewidth
            \vtop{\raggedright\hyphenpenalty\z@\exhyphenpenalty\z@
                \doublehyphendemerits\z@\finalhyphendemerits\z@
                \strut ##1\strut}%
        }%
        % If the linebreak is at a space, the latter will be displayed as visible
        % space at end of first line, and a continuation symbol starts next line.
        % Stretch/shrink are however usually zero for typewriter font.
        \def\FV@Space {%
            \nobreak\hskip\z@ plus\fontdimen3\font minus\fontdimen4\font
            \discretionary{\copy\Wrappedvisiblespacebox}{\Wrappedafterbreak}
            {\kern\fontdimen2\font}%
        }%

        % Allow breaks at special characters using \PYG... macros.
        \Wrappedbreaksatspecials
        % Breaks at punctuation characters . , ; ? ! and / need catcode=\active
        \OriginalVerbatim[#1,codes*=\Wrappedbreaksatpunct]%
    }
    \makeatother

    % Exact colors from NB
    \definecolor{incolor}{HTML}{303F9F}
    \definecolor{outcolor}{HTML}{D84315}
    \definecolor{cellborder}{HTML}{CFCFCF}
    \definecolor{cellbackground}{HTML}{F7F7F7}

    % prompt
    \makeatletter
    \newcommand{\boxspacing}{\kern\kvtcb@left@rule\kern\kvtcb@boxsep}
    \makeatother
    \newcommand{\prompt}[4]{
        {\ttfamily\llap{{\color{#2}[#3]:\hspace{3pt}#4}}\vspace{-\baselineskip}}
    }
    

    
    % Prevent overflowing lines due to hard-to-break entities
    \sloppy
    % Setup hyperref package
    \hypersetup{
      breaklinks=true,  % so long urls are correctly broken across lines
      colorlinks=true,
      urlcolor=urlcolor,
      linkcolor=linkcolor,
      citecolor=citecolor,
      }
    % Slightly bigger margins than the latex defaults
    
    \geometry{verbose,tmargin=1in,bmargin=1in,lmargin=1in,rmargin=1in}
    
    

\begin{document}
    
    \maketitle
    
    

    
    

    \section{Homework 6 (Due 21 Mar)}\label{homework-6-due-21-mar}

\textbf{Due 21 Mar (midnight)}

Total points: 100

    \begin{tcolorbox}[breakable, size=fbox, boxrule=1pt, pad at break*=1mm,colback=cellbackground, colframe=cellborder]
\prompt{In}{incolor}{5}{\boxspacing}
\begin{Verbatim}[commandchars=\\\{\}]
\PY{k+kn}{import}\PY{+w}{ }\PY{n+nn}{numpy}\PY{+w}{ }\PY{k}{as}\PY{+w}{ }\PY{n+nn}{np}
\PY{k+kn}{from}\PY{+w}{ }\PY{n+nn}{math}\PY{+w}{ }\PY{k+kn}{import} \PY{o}{*}
\PY{k+kn}{import}\PY{+w}{ }\PY{n+nn}{matplotlib}\PY{n+nn}{.}\PY{n+nn}{pyplot}\PY{+w}{ }\PY{k}{as}\PY{+w}{ }\PY{n+nn}{plt}
\PY{k+kn}{import}\PY{+w}{ }\PY{n+nn}{pandas}\PY{+w}{ }\PY{k}{as}\PY{+w}{ }\PY{n+nn}{pd}
\PY{o}{\PYZpc{}}\PY{k}{matplotlib} inline
\PY{n}{plt}\PY{o}{.}\PY{n}{style}\PY{o}{.}\PY{n}{use}\PY{p}{(}\PY{l+s+s1}{\PYZsq{}}\PY{l+s+s1}{seaborn\PYZhy{}v0\PYZus{}8\PYZhy{}colorblind}\PY{l+s+s1}{\PYZsq{}}\PY{p}{)}
\end{Verbatim}
\end{tcolorbox}

    \subsection{Introduction to Homework
6}\label{introduction-to-homework-6}

This week's exercises focus on oscillators and how to approximate the
solution to the equations of motion using the SHO. The relevant reading
background is: 1. chapter 5 of Taylor

\begin{enumerate}
\def\labelenumi{\arabic{enumi}.}
\setcounter{enumi}{1}
\item
  chapter 6.6-6.8 of Boas
\item
  chapters 2.0-2.3 and 5.0-5.2 of Strogatz
\end{enumerate}

In both textbooks there are many nice worked out examples.

\subsubsection{Practicalities about homeworks and
projects}\label{practicalities-about-homeworks-and-projects}

\begin{enumerate}
\def\labelenumi{\arabic{enumi}.}
\item
  You can work in groups (optimal groups are often 2-3 people) or by
  yourself. If you work as a group you can hand in one answer only if
  you wish. \textbf{Remember to write your name(s)}!
\item
  Homeworks are available ten days before the deadline.
\item
  How do I(we) hand in? You can hand in the paper and pencil exercises
  as a \textbf{single scanned PDF document}. For this homework this
  applies to exercises 1-5. Your jupyter notebook file should be
  converted to a \textbf{PDF} file, attached to the same PDF file as for
  the pencil and paper exercises. All files should be uploaded to
  Gradescope.
\end{enumerate}

\textbf{\href{../resources/gradescope-submissions.md}{Instructions for
submitting to Gradescope}.}

    \subsubsection{Exercise 1 (10pt) Morse Potential as an
SHO}\label{exercise-1-10pt-morse-potential-as-an-sho}

If the potential has a local minimum, we can often find SHO
approximation for that potential near the local minimum.

The \href{https://en.wikipedia.org/wiki/Morse_potential}{Morse
potential} is a convenient model for the potential energy of a diatomic
molecule. The potential is a radial one and thus one-dimensional. It is
given by,

\[U(r) = A\left[ \left(e^{(R-r)/S}-1\right)^2-1\right]\]

where the distance between the centers of the two atoms is \(r\), and
the constants \(A\), \(R\), and \(S\) are all positive. Here \(S<<R\).

\begin{itemize}
\tightlist
\item
  1a (2pt) Sketch (or plot) the potential as a function of \(r\).
\item
  1b (3pt) Find the equilibrium position of the potential, i.e.~the
  position where the potential is at a minimum. We will call this
  \(r_e\).
\item
  1c (3pt) Rewrite the potential in terms of the displacement from
  equilibrium, \(r = r_e + x\). Expand the potential to second order in
  \(x\).
\item
  1d (2pt) Find the effective spring constant, \(k\), for the potential
  near the minimum. What is the frequency of small oscillations about
  the minimum?
\end{itemize}

    \subsubsection{Exercise 2 (10pt), Time Averaging and the
SHO}\label{exercise-2-10pt-time-averaging-and-the-sho}

Time Averaging is a common tool to use with periodic systems. It also us
to discuss what happens to different properties of the system over one
period.

An SHO has a period \(\tau\). We can find the time average of a variable
\(f(t)\) over one period, \(\langle f \rangle\), by averaging over the
period,

\[\langle f \rangle = \frac{1}{\tau}\int_0^\tau f(t)dt\]

\begin{itemize}
\tightlist
\item
  2a (2pt) Show that the time average of the position of the SHO is
  zero.
\item
  2b (2pt) Show that the time average of the velocity of the SHO is
  zero.
\item
  2c (6pt) Show that the time average of the kinetic energy of the SHO
  is equal to the time average of the potential energy (and importantly,
  non-zero). If the total energy is \(E\), these time averages are equal
  to \(E/2\). You might need to show that the very useful trigonometric
  identity,
\end{itemize}

\[\langle \sin^2(\omega t-\delta)\rangle = \langle \cos^2(\omega t-\delta)\rangle = \frac{1}{2}\]

    \subsubsection{Exercise 3 (10pt), Toy
Potential}\label{exercise-3-10pt-toy-potential}

Consider a toy potential of the form,

\[U(r) = U_0\left(\dfrac{r}{R}+\lambda^2\frac{R}{r}\right)\]

where \(U_0\), \(R\), and \(\lambda\) are all positive constants and the
domain of the potential is \(0<r<\infty\).

\begin{itemize}
\tightlist
\item
  3a (2pt) Sketch (or plot) the potential as a function of \(r\).
\item
  3b (3pt) Find the equilibrium position of the potential, i.e.~the
  position where the potential is at a minimum. We will call this
  \(r_e\).
\item
  3c (5pt) Rewrite the potential in terms of the displacement from
  equilibrium, \(r = r_e + x\). Expand the potential to second order in
  \(x\). What is the effective spring constant, \(k\), for the potential
  near the minimum? What is the frequency of small oscillations about
  the minimum?
\end{itemize}

    \subsubsection{Exercise 4 (10pt), Defining
Periodicity}\label{exercise-4-10pt-defining-periodicity}

A common issue with oscillators is determining their periodicity. For
the SHO, we can show that the period is \(2\pi/\omega_0\) where
\(\omega_0\) is the natural frequency of the SHO. How might we define a
periodicity more generally? Let's start with the damped harmonic
oscillator. Consider a weakly damped oscillator (\(\beta < \omega_0\)).
The motion of the oscillator is given by,

\[x(t) = A e^{-\beta t}\cos(\omega_1 t - \delta)\]

where \(A\) is the amplitude, \(\beta\) is the damping constant,
\(\omega_1 = \sqrt{\omega_0^2 - \beta^2}\) and \(\delta\) is the phase.

The motion decays with time, but we can still define a periodicity,
\(\tau_1\), which is the time between peaks in the motion.

\begin{itemize}
\tightlist
\item
  4a (4pt) Sketch (or plot) the motion of the oscillator as a function
  of time. Show that you can find \(\tau_1 = 2\pi/\omega_1\) from
  looking at successive maxima.
\item
  4b (3pt) Show that an equivalent definition of \(\tau_1\) is twice the
  time between zero crossings of the motion.
\item
  4c (3pt) If the damping \(\beta\) is half the natural frequency
  (\(\omega_0\)), how does the amplitude of the motion decay in one
  period?
\end{itemize}

    \subsubsection{Exercise 5 (10pt), Planning your Final
Project}\label{exercise-5-10pt-planning-your-final-project}

The point of this exercise is to get you thinking about what you want to
do for a final project and how you will accomplish it. Research and
project planning are important skills to develop and it takes time to do
so. We are trying to scaffold that development and give you the
opportunity to practice these skills. So, you will be responsible for
developing a plan for your final project, and the timeline under which
you will accomplish it. Our job is to provide feedback and guidance on
your plan, and to offer support in the development and construction of
your final project.

\textbf{Remember there is no right or wrong way to do this, it is about
developing a plan that works for you.}

Your final project is your chance to bring all the tools we have
developed to bear on a question that interests you. You have reviewed
the information on computational essays before, but here it is again in
case it's useful for this exercise:

\begin{enumerate}
\def\labelenumi{\arabic{enumi}.}
\tightlist
\item
  Wolfram's
  \href{https://writings.stephenwolfram.com/2017/11/what-is-a-computational-essay}{What
  is a Computational Essay?}
\item
  Tor and my short paper:
  \href{https://arxiv.org/abs/1909.12697}{Computational Essays: An
  Avenue for Scientific Creativity in Physics}
\item
  Wolfram's
  \href{https://www.wolframcloud.com/obj/Expositions/Published/ComputationalEssayGuidelines}{Steps
  to Writing a Computational Essay}
\item
  \href{https://uio-ccse.github.io/computational-essay-showroom}{University
  of Oslo's Computational Essay Showroom}
\end{enumerate}

For your final project, you may work alone, in pairs, or in groups of up
to three people. The expectation for your work will be higher if you
have more people in your group; your explorations will need to be more
comprehensive and detailed, and your essay will be expected to be longer
and more polished if you have more people in your group. You can also
develop more interesting ways of presenting your work that demonstrate
the creativity and depth of your explorations if you have more people in
your group. If you have any questions, just ask.

For this exercise, we want you to get started on planning your final
project by sending us a brief proposal with a timeline. To help you, we
have provided a schedule where you can see the checkpoints for your
project, and when we will have a comprehensive work week. You should
take advantage of the work week to make significant progress on your
project; especially if there are technical aspects you are unsure about.

\paragraph{Schedule for Final Project}\label{schedule-for-final-project}

\begin{itemize}
\tightlist
\item
  28 March - HW 7 will have a request for your 1st project update.
\item
  11 Apr - Midterm Project 2 will have a request for your 2nd project
  update.
\item
  18 Apr - HW 8 will have a request for your 3rd project update.
\item
  Week of 20-24 Apr - Project work week in-class.
\item
  28 Apr - Final project due.
\end{itemize}

\paragraph{Your Proposal}\label{your-proposal}

Your proposal should be a one to two page, single spaced, document that
outlines the following:

\begin{enumerate}
\def\labelenumi{\arabic{enumi}.}
\tightlist
\item
  The question you are interested in exploring. The motivation and
  interest to you and your group mates should be made clear. That is,
  motivate the question, why is it interesting?
\item
  References that provide background for this work. This should include
  the sites, resources, and references you have consulted to get
  started.
\item
  The approach you think you need to take to answer this question.
  Include the concepts and tools you plan to use to explore the
  question. That is, what have we developed in class that you will use
  to explore the question?
\item
  The timeline for your project. This should include the checkpoints in
  the schedule above, and any additional milestones you think are
  important. Consider making a
  \href{https://en.wikipedia.org/wiki/Gantt_chart}{Gantt chart} to help
  you plan your project. This is a useful tool for planning and tracking
  your progress. It's almost expected for proposals in science and
  engineering. Here, We suggest you take advantage of the work week to
  make significant progress on or to polish your project.
\end{enumerate}

\paragraph{Your writing}\label{your-writing}

You should consider this a formal proposal. It should be well written,
clear, and concise. You should also consider the audience for this
proposal. You should write this proposal for someone who is not in this
class, but does know physics. Think about writing this to another
student, a graduate student, or a professor. Someone that has a good
understanding of physics, but not necessarily the details of your
physics. You need to explain the motivation, the physics background, the
approach, and the timeline in a way that is compelling and clear.

Your proposal need not be long, but it should be complete. Here are some
additional resources for you to develop a good proposal:

\begin{itemize}
\tightlist
\item
  \href{https://pmc.ncbi.nlm.nih.gov/articles/PMC5037942/}{NIH, How to
  write a research proposal?}
\item
  \href{https://www.sheffield.ac.uk/study-skills/research/methods/proposal}{University
  of Sheffield, How to write a research proposal}
\end{itemize}

\textbf{Reference pages do not count towards the length, so if you have
a lot of references, that's fine. But the main text should be one to two
pages. This is how it works with most proposals in science and
engineering.}

    \subsubsection{Exercise 6 (50 pt), Find your own 1D
Oscillator}\label{exercise-6-50-pt-find-your-own-1d-oscillator}

We have built all the tools to study 1D unforced oscillators. Now you
get to pick your own potential and study it. You can pick any 1D
potential you like, but it should have a local minimum. Make sure it is
not a driven oscillator (i.e., no explicit time dependence in the
equations of motion). To earn full credit for this exercise, you must:

\begin{itemize}
\tightlist
\item
  6a (5pt) Present the potential and describe its origin, why it is
  interesting, where it comes from, etc. Educate us about it.
\item
  6b (5pt) Sketch (or plot) the potential as a function of it's argument
  (and chosen variables) and find the equilibrium position of the
  potential, i.e.~the position where the potential is at a minimum.
\item
  6c (10pt) Rewrite the potential in terms of the displacement from
  equilibrium. Expand the potential to second order to find the
  effective spring constant, \(k\), for the potential near the minimum.
  What is the frequency of small oscillations about the minimum?
\item
  6d (10pt) Construct the equations of motion for the potential and
  solve them numerically. Choose initial conditions and parameters that
  give oscillatory motion. Note it doesn't have to be SHO (In fact, it
  probably won't be). Plot the position as a function of time. Make sure
  we can see the oscillations.
\item
  6e (10pt) Plot the phase diagram of the trajectory (you don't have to
  produce a phase diagram, but just plot the trajectory in phase space).
  What does the phase diagram tell you about the motion?
\item
  6f (10pt) Find the period of your motion. Here you might have to make
  some definitions of what periodicity means for your potential.
\end{itemize}

\textbf{Note: this might seem similar to your midterm, but notice we
expect you to do some research on the potential in 6a, and we are going
into more depth with questions 6e and 6f. This is also a scaffold for
your final project in terms of practicing aspects that should appear.}

\paragraph{Examples of 1D potentials}\label{examples-of-1d-potentials}

\subparagraph{Simple Pendulum (Nonlinear Small Angle
Approximation)}\label{simple-pendulum-nonlinear-small-angle-approximation}

\[ V(\theta) = mgh(1 - \cos(\theta)) \] where \(m\) is the mass, \(g\)
is the acceleration due to gravity, \(h\) is the length of the pendulum,
and \(\theta\) is the angular displacement.

\subparagraph{Nonlinear Spring (Hardening or
Softening)}\label{nonlinear-spring-hardening-or-softening}

\[ V(x) = \frac{k}{2} x^2 + \frac{\beta}{3} x^3 \] where \(k\) and
\(\beta\) are constants. Depending on the sign of \(\beta\), the spring
can exhibit hardening (\(\beta > 0\)) or softening (\(\beta < 0\))
nonlinearity.

\subparagraph{Lennard-Jones Potential Oscillator (for a diatomic
molecule
model):}\label{lennard-jones-potential-oscillator-for-a-diatomic-molecule-model}

\[ V(r) = 4\epsilon \left[ \left(\frac{\sigma}{r}\right)^{12} - \left(\frac{\sigma}{r}\right)^6 \right] \]
where \(\epsilon\) is the depth of the potential well, \(\sigma\) is the
finite distance at which the inter-particle potential is zero, and \(r\)
is the distance between particles.

\subparagraph{Morse Potential (for molecular
vibrations)}\label{morse-potential-for-molecular-vibrations}

\[ V(x) = D_e \left(1 - e^{-a(x - x_0)}\right)^2 \] where \(D_e\) is the
depth of the potential well, \(a\) is a constant related to the width of
the well, \(x\) is the displacement from equilibrium, and \(x_0\) is the
equilibrium bond length. This potential models the energy of a diatomic
molecule as a function of the distance between atoms, showing
oscillatory behavior that represents molecular vibrations.

\subparagraph{Double Well Potential}\label{double-well-potential}

\[ V(x) = -\frac{\mu}{2} x^2 + \frac{\lambda}{4} x^4 \] where \(\mu\)
and \(\lambda\) are positive constants. This system exhibits bistability
with two stable equilibria, leading to interesting nonlinear dynamics
and potential oscillations between the wells under certain conditions.

    \subsubsection{Extra Credit - Integrating Classwork With
Research}\label{extra-credit---integrating-classwork-with-research}

This opportunity will allow you to earn up to 5 extra credit points on a
Homework per week. These points can push you above 100\% or help make up
for missed exercises. In order to earn all points you must:

\begin{enumerate}
\def\labelenumi{\arabic{enumi}.}
\item
  Attend an MSU research talk (recommended research oriented Clubs is
  provided below)
\item
  Summarize the talk using at least 150 words
\item
  Turn in the summary along with your Homework.
\end{enumerate}

Approved talks: Talks given by researchers through the following clubs:
* Research and Idea Sharing Enterprise (RAISE)\hspace{0pt}: Meets
Wednesday Nights Society for Physics Students (SPS)\hspace{0pt}: Meets
Monday Nights

\begin{itemize}
\item
  Astronomy Club\hspace{0pt}: Meets Monday Nights
\item
  Facility For Rare Isotope Beam (FRIB) Seminars: \hspace{0pt}Occur
  multiple times a week
\end{itemize}

    


    % Add a bibliography block to the postdoc
    
    
    
\end{document}
